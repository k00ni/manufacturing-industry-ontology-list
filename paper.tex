\documentclass{article}
\usepackage[utf8]{inputenc}
\usepackage[english]{babel}
\usepackage{caption}
\usepackage{float}
\usepackage{subcaption}
\usepackage[table,xcdraw]{xcolor}

\usepackage[lmargin = {2.7cm},
rmargin = {2.7cm},
tmargin = {2.2cm},
bmargin = {2.2cm}
]{geometry}

\usepackage{comment}
\usepackage[hidelinks]{hyperref}
\usepackage[capitalise]{cleveref}

% TexStudio build config line: txs:///bibtex | txs:///biber | txs:///compile | txs:///view
\usepackage[
backend=biber,
style=numeric,
sorting=nty
]{biblatex}
\addbibresource{paper.bib}

\usepackage{graphicx}
\graphicspath{ {./} }

\title{Ontologien aus dem Bereich der Industrie (TODO)}
\author{\textbf{Konrad Abicht} \\ k.abicht@gmail.com}
\date{TODO}

\begin{document}

\maketitle

\begin{abstract}
    TODO
\end{abstract}

\section{Einleitung}

Im Folgenden werden die Ergebnisse einer systematischen Internet- und Literaturrecherche zur Ermittlung aller öffentlich zugänglichen\footnote{RDF-Dokument ist über eine URL abrufbar.} und in einer formalen Sprache (z.B. RDF/OWL) vorliegenden Ontologien (und Vokabulare) vorgestellt, die sich thematisch im Bereich Industrie einordnen lassen.
Das Ziel war die Erstellung einer Übersicht, bestehend aus maschinenlesbaren Ontologien, die eine automatisierte Nutzung des modellierten Wissens erlauben (z.B. OWL Reasoning oder Inferenz).

[...]

Die Arbeit ist wie folgt strukturiert: [...] Kapitel 2 enthält ... in Kapitel 3 folgt ... usw.

\subsection{Öffentlich verfügbare Forschungsdaten}

Die erstellten Forschungsdaten stehen in einem Github-Repository für die Öffentlichkeit zur Verfügung:

\textbf{TODO:} Link einfügen

\textbf{TODO:} Die Inhalte unterliegen dabei den Bedingungen der Creative Commons BY 4.0.

\section{Fachlicher Hintergrund}

In diesem Kapitel werden die notwendige Vorkenntnise für diese Arbeit kurz zusammengefasst.
Ergänzend dazu werden weitere Begriffe informell definiert, weil die Rechercheergebnisse zeigen, dass Ontologie-Autoren keine einheitlichen Begriffe für die Einordnung ihrer Arbeiten verwenden.

\subsection{Semantic Web und formale Sprachen}

\textbf{TODO}
RDF/OWL
KIF

\subsection{Ontologien und Kontrollierte Vokabulare}

Im Rahmen dieser Recherche lag der Fokus auf einer Gruppe von Ontologien.
Zur Beschreibung dieser Gruppe wurde die Definition von Fabian Neuhaus \cite{neuhaus2018ontology} genutzt, jedoch mit ein paar Ergänzungen.
Zusammengenommen ist eine Ontologie Teil der Gruppe, wenn sie alle folgenden Punkte erfüllt:

\begin{enumerate}
    \item Die Ontologie besitzt eine URL zu einer Textdatei, welche in einer formalen Sprache (RDF/OWL oder KIF) vorliegt.
    \item Die Ontologie stellt ein Vokabular zur Beschreibung des Fachgebietes bereit.
    \item Der Ontologie liegt eine logische Theorie (z.B. in Form von Axiomen, Regeln, Hierarchien) über das Fachgebiet zugrunde, die auf das Vokabular zurückgreift.
\end{enumerate}

Im Rahmen der Recherche wurden Publikationen zu Ontologien gefunden, bei denen die öffentlich zugängliche Daten nicht ermittelt werden konnten.
Diese Arbeiten wurden aus den folgenden Gründen ignoriert:
\begin{enumerate}
    \item Ohne die zugehörigen Rohdaten in formaler Sprache kann nicht sichergestellt werden, dass eine Ontologie vollständig vorliegt. Davon abgeleitete Arbeiten könnten zu unerwarteten Fehlern und Widersprüchen führen. Weiterhin würde man erzwingen, dass das beschriebenen Wissen zuerst manuell in eine entsprechend Form gebracht werden muss, um eine Weiterverarbeitung zu ermöglichen.
    \item Eine automatisierte Nutzung des modellierten Wissens wird als nicht gegeben angesehen, wenn das Wissen selbst nicht in einer maschinenlesbaren, formalen Sprache vorliegt. Publikationen zu Ontologien liegen häufig als PDF-Datei vor, wobei das darin beschriebene Faktenwissen (inkl. Formeln in Description Logic und Prädikatenlogik 1. Stufe) nicht entsprechend annotiert ist. Für einen Parser sieht es daher genauso aus, wie der restliche Inhalt.
\end{enumerate}

Im Kontext dieser Arbeit bezeichnet der Begriff Ontologie auch Vokabulare.
Der Grund für diese Festlegung sind die Rechercheergebnisse, die zeigen, dass die Autoren häufig ihre eigenen Arbeiten entweder als Ontologie oder Vokabular, oder beides, bezeichneten\footnote{Zum Beispiel bezeichnet Martin Hepp GoodRelations als standardisiertes Vokabular für Produkte, Preise und Unternehmensdaten, nutzt jedoch Ontologie als Quasi-Synonym, Zitat: "GoodRelations is a standardized vocabulary (also known as "schema", "data dictionary", or "ontology") for product, price, store, and company data that can [...]", Quelle: \url{https://www.heppnetz.de/ontologies/goodrelations/v1.html}}.

\subsection{Ontologiearten}

\textbf{TODO: baue Grafik mit Darstellung der Arten und Mapping der Dreiteilung und Fünfteilung}

Die Rechercheergebnisse zeigten, dass Autoren ihre Ontologien überwiegend nach der etablierten Dreiteilung Top-Level-, Kern- und Task-Ontologie einordnen.
An der Spitze der Hierarchie steht die \textbf{Top-Level Ontologie} (Synonyme: Upper Ontology, Foundational Ontology).
Sie beinhaltet fachgebiets-unabhängige Inhalte\footnote{Begriffe, Relationen, Regeln, Axiome etc.} zur Beschreibung des Ausschnitts der Realität.
Zu den bekanntesten gehören Suggested Upper Merged Ontology (SUMO), Descriptive Ontology for Linguistic and Cognitive Engineering (DOLCE) und Basic Formal Ontology (BFO)\footnote{Mehr Details zu diesen in einem späteren Kapitel.}.
Top-Level-Ontologien bringen häufig eine eigene Theorie zur Beschreibung der Realität mit.
Nutzen abgeleitete Ontologien die Inhalte einer Top-Level Ontologie, so übernehmen sie implizit auch die zugehörige Theorie.

In der genannte Hierarchie folgt unter der Top-Level Ontologie die \textbf{Kern-Ontologie} (Synonyme: Core Ontology, Domain Ontology).
Ihr Inhalt bezieht sich entweder stärker auf ein Fachgebiet oder ergänzt die Inhalte anderer Ontologien.

Unter den Kern-Ontologien folgt die \textbf{Task-Ontologie} (Synonyme: Application Ontology).
Sie nutzt in der Regel die Inhalte einer Top-Level Ontologie und sowie mehrerer Kern-Ontologien und stellen eigene Inhalte für einen konkreten Anwendungsfall bereit.

Diese Dreiteilung wird von manchen Ontologie-Autoren nicht verwendet (z.B. \cite{kulvatunyou2022}).
Sie verfeinern den "Mittelbau", die oben genannten Kern-Ontologien, und unterscheiden zwischen Mid-level und Domain-Ontologien.
Diese Aufgliederung hat ihre Berechtigung und kann aus vielen Gründen naheliegend sein.
Jedoch existiert bisher keine Methode zur eindeutigen inhaltlichen Abgrenzung von Mid-level- und Domain-Ontologien, weshalb diese Arbeit nur die Dreiteilung nutzt.
Mid-level und Domain-Ontologien werden daher bei den Kern-Ontologien eingruppiert.
Damit ist eine einheitliche Einteilung der recherchierten Ontologien möglich.

\subsection{intelligent information Request and Delivery Standard (iiRDS)}

"iiRDS, the standard for intelligent information request and delivery, offers a metadata model for the domain of technical documentation and enables the exchange of documentation content enriched with metadata between companies, platforms, and services. iiRDS can be combined with ontologies describing product features and components as well as events in smart factories. It thus allows applications, for example, to deliver relevant information for installation, maintenance or troubleshooting tasks triggered by events or machine states."

https://iirds.org/fileadmin/iiRDS\_specification/20201103-1.1-release/index.html

\url{https://data.ontocommons.linkeddata.es/vocabulary/Iirds%E2%80%93IntelligentInformationRequestAndDeliveryStandard}

TODO: Passt das?

\subsubsection{Industry Foundation Classes (IFC)}

"The Industry Foundation Classes IFC represent an open specification for Building Information Modeling BIM data that is exchanged and shared among the various participants in a building construction or facility management project. IFC's are the international openBIM standard."

https://standards.buildingsmart.org/IFC/RELEASE/IFC4/ADD1/HTML/

\section{Verwandte Arbeiten (Related Work)}

- Welche Untersuchungen wurden bereits dahingehend unternommen?

- Wozu gab es noch keine oder nur unzureichende Untersuchungen?

\subsection{Ontologies for Industry 4.0}
%
% Ontologies for Industry 4.0
% https://www.cambridge.org/core/services/aop-cambridge-core/content/view/BF86BB5310356D642C82470D67974804/S0269888919000109a.pdf/div-class-title-ontologies-for-industry-4-0-div.pdf

Zu Beginn der Publikation "Ontologies for Industry 4.0" \cite{kumar2019ontologies} (2019) geben die Autoren zuerst eine kurze Einordnung des Themengebietes Industrie 4.0, Factory 4.0 und Smart Manufacturing.
Diese wird ergänzt durch eine historische Einordnung.
Danach folgt eine Zusammenfassung relevanter Herausforderungen (z.B. Mensch-Maschine-Kommunikation oder Datenanalyse) und die zur Bewältigung genutzten Technologien (z.B. Internet of Things, 5G, Cloud).
Im zweiten Teil der Arbeit wird der ontologische Rahmen vorgestellt und die Autoren listen eine Reihe von Anwendungsbereichen für Ontologien auf, jedoch ohne diese näher einzuordnen.
Immerhin wird am Ende des Abschnitts ein kurzer Vergleich zwischen zwei verschiedenen Strukturierungsansätzen für Ontologien andiskutiert.
Im Kapitel 2.2 stellen die Autoren einige Resultate ihrer Standardisierungsvorhaben zu Industrieontologien vor und beschreiben den aktuellen Stand.
Das Hauptaugenmerk lag dabei zuerst auf der Kommunikation zwischen Robotern, menschlichen Bedienern, Endkunden und diversen Zulieferern.
Laut Aussage der Autoren sollten diese Themen zuerst geklärt werden, um daraus resultierende Probleme zu vermeiden.
Ihre Arbeit basiert auf dem IEEE-Beitrag namens "IEEE 1872-2015 Standard Ontologies for Robotics and Automation"\footnote{1872-2015 - IEEE Standard Ontologies for Robotics and Automation: \url{https://ieeexplore.ieee.org/document/7084073}}.
Diese Ontologien bieten eine ontologische Basis für weiterführende Konzepte aus dem Bereich Industrie 4.0.
Danach folgt eine kurze Vorstellung der folgenden Ontologien:
\begin{enumerate}
    \item \textbf{CORA:} Core Ontology for Robotics and Automation
    \item \textbf{ROA:} The Ontology for Autonomous Robotics\footnote{In der zugehörigen Publikation\cite{olszewska2017ontology} kürzen die Autoren die Ontologie jedoch mit ORA ab.}
    \item \textbf{ORArch}: Ontology for Robotic Architecture
    \item \textbf{O4I4}: Ontology for Industry 4.0
\end{enumerate}

Im dritten Teil der Arbeit gehen die Autoren die folgenden Szenarien im Bereich Industrie 4.0 ein: Smart-rapid prototyping scenario und UAV’s good delivery scenario.
Jedes Szenario wird kurz eingeführt und danach der Einsatz der Ontologien beschrieben.
Es ist erwähnenswert, dass die Publikation an mehreren Stellen eine Auflistung von relevanten Publikationen enthält, die jedoch nur benannt aber nicht inhaltlich eingeordnet werden. Damit bieten die Autoren einen guten Ausblick, jedoch fehlt dem Leser die Einordnung in das Papier.


\subsection{Where to Publish and Find Ontologies? A Survey of Ontology Libraries}

Die Autoren des Papiers "Where to Publish and Find Ontologies? A Survey of Ontology Libraries" \cite{d2012publish} (Jahr 2012) geben einen Überblick über Ontologie-Bibliotheken.
Es wurden verschiedene Bezeichnungen\footnote{Ontology Directory, Ontology Repository, Ontology Archive} für solche Systeme gefunden, weshalb man sich zum besseren Verständnis auf Ontologie-Bibliothek festgelegt hat.
Diese Publikation ist für diese Arbeit relevant, weil sie einen Überblick vorhandener Ontologie-Bibliotheken gibt, worüber man Ontologien beziehen kann.
Neben Metadaten wie Titel, Lizenz und die neuste Version, werden häufig auch die zugehörigen RDF Daten bereitgestellt.
Die Autoren stellen jedoch fest, dass viele bekannte Ontologie-Bibliotheken nicht mehr existieren bzw. nicht weiterbetrieben werden.
Wir haben die Webseiten der genannten Systeme selbst geprüft und konnten die folgenden weiterhin abrufen:

\begin{enumerate}
    \item BioPortal (\url{https://bioportal.bioontology.org/}, Themengebiete: Biomedizin)
    \item OBO Foundry (\url{https://obofoundry.org/}, Themengebiete: Biologie und Biomedizin)
    \item oeGOV (\url{http://www.oegov.us/}, Themengebiet: e-Government)
    \item Ontology Lookup Service (\url{https://www.ebi.ac.uk/ols4}, Themengebiete: Biomedizin)
    \item Ontology Design Patterns (\url{http://ontologydesignpatterns.org/wiki/Main\_Page}, viele Themengebiete, siehe auch \url{http://ontologydesignpatterns.org/wiki/Community:Domain})
    \item ONKI ontology server (\url{https://onki.fi/en/}, verschiedene Themengebiete)
\end{enumerate}

Nach der Vorstellung vorhandener Ontologie-Bibliohteken stellen verschiedene Charakteristiken dieser Systeme vor.
Daneben gibt es auch Angaben über die konkreten ontologischen Inhalte der Systeme (z.B. Ontology-Mappings, Ontology-Level Relations).
Eine wichtige Erkenntis war, dass die Systeme in der Regel nur einen Ausschnitt an Ontologien anbieten.
Das haben wir selbst bei unserer Recherche festgestellt.
Für die Ermittlung passender Ontologien eines konkreten Fachgebietes ist man gezwungen, mehrere Ontologie-Bibliotheken zu konsultieren, wobei ergänzende Suchen mittels einer Suchmaschine (wie Google) nicht ausbleiben.
Am Ende der Publikation weisen die Autoren noch auf eine Reihe von offenen Problemen solcher Systeme hin.
Als ein Problem wurden die oft mangelhaften Angaben zur Weiterverwendung und Lizensierung von Ontologien genannt.
Ein anderes die oft mangelnde Unterstützung für die Mitarbeit (z.B. Melden von Fehlern, Einreichung von Ergänzungen).
Bei unserer Recherche sind wir auch auf diese Probleme gestoßen.

\section{Methodik}

Das Ziel der Literatur- und Internetrecherche war die Erstellung einer Übersicht, bestehend aus maschinenlesbaren Ontologien, die eine automatisierte Nutzung des modellierten Wissens erlauben (z.B. OWL Reasoning oder Inferenz).
Das Ergebnis sollte dabei die folgenden Forschungsfragen beantworten:

\begin{enumerate}
    \item Welche Ontologien für das Fachgebiet Industrie gibt es?
    \item Welche dieser Ontologien werden aktiv betreut bzw. wann war die letzte dokumentierte Aktivität im Projekt?
    \item Sollten Top-Level-Ontologien von diesen Ontologien genutzt werden, welche sind das?
\end{enumerate}

Diese Forschungsfragen ermöglichen einen neutralen Blick auf die vorhandenen Ontologien.

Bei der gesamten Recherche waren sowohl deutsch- als auch englischsprachige Inhalte von Interesse.

\subsection{Literaturrecherche}

Viele Ontologien wurden von Mitgliedern der wissenschaftlichen Gemeinschaft entwickelt.
Daher stand die Literaturrecherche am Anfang, um die Publikationen zu finden, welche eine Ontologie und ihre Inhalte näher vorstellen.
Die Art der Publikationen reichen von wissenschaftlichen Beiträgen über White-Paper bis zu einfachen Readme-Dateien.
Als Hauptanlaufpunkt wurde Google Scholar eingesetzt.
Es wurden die Suchergebnisse für jeden dieser Suchbegriffe näher untersucht, insofern sie von einer Ontologie handelten.
In der Regel basierte eine vorgestellte Ontologie auf weiteren Ontologien, welche in einem separaten Kapitel vorgestellt werden.

Es wurden ausschließlich englischsprachige Suchbegriffe verwendet, die relevant für das Fachgebiet Industrie sind.
Im Folgenden ist eine Auflistung aller genutzter Suchbegriffe:

\subsubsection{Industry Ontology}

Eine damit gekennzeichnete Publikation beschreibt eine Ontologie mit direktem Bezug zur Industrie.

\subsubsection{Industry 4.0}

Dieser Begriff bezeichnet die vierte industrielle Revolution, bezogen auf den technologischen Fortschritt. In einem I4.0-Szenario steht der Fertigungsprozess im Mittelpunkt und bildet die Hauptaktivität, bei der Technologien und Hilfsmittel wie autonome Roboter, Cyber-Physical-Systems, Big-Data Analysis und IoT eingesetzt werden.

\subsubsection{Manufacturing, Smart Manufacturing, Smart Factory, Factory 4.0}

Ein Oberbegriff für eine (evolutionäre) Initiative zur Änderung der Geschäftsstrategie im Fertigungsbereich.
Hierbei sollen Fertigungszentren modernisiert werden unter Einbeziehung von Technologien wie Industry Internet of Things (IIOT), 3D-Modeling, Industrieautomation, Mobile-Computing und Intelligente analysebasierte Entscheidungssysteme.
In diesem Zusammenhang spielen auch die Themen wie Wartung (Maintenance) eine wichtige Rolle.

\subsubsection{Cyber-Physical Systems (CPS)}

Cyber-physische Systeme sind Systeme, bei denen informations- und softwaretechnische mit mechanischen Komponenten verbunden sind, wobei Datentransfer und -austausch sowie Kontrolle bzw. Steuerung über ein Netzwerk (z.B. Internet) in Echtzeit erfolgen.
Wesentliche Bestandteile sind mobile und bewegliche Einrichtungen, Geräte und Maschinen (darunter auch Roboter), eingebettete und vernetzte Systeme (Internet der Dinge).
Sensoren registrieren und verarbeiten Daten aus der physikalischen Welt, Aktoren (Antriebselemente) wirken auf die physikalische Welt ein, sodass z.B. Weichen gestellt, Schleusen geöffnet, Fenster und Türen geschlossen, Produktionsvorgänge begonnen, geändert und angehalten werden \cite{GablerCPS2024}.

\subsubsection{Supply Chain Management}

Supply Chain Management (= Lieferkettenmanagement) ist eine Bezeichung, die seit dem Jahr 1990 verstärkt in der Literatur verwendet wird. Eine Lieferkette (Supply Chain) kann definiert werden als "eine Menge von mindestens drei unabhängig voneinander agierenden Firmen (oder Personen), die direkt an den vor- und nachgelagerten Strömen von Produkten, Dienstleistungen, Finanzmitteln und/oder Informationen von einer Quelle zu einem Kunden beteiligt sind" (\cite{mentzer2001defining}, S. 4). Aufgrund der Vielfalt an Definitionen für Lieferkettenmanagement (Supply Chain Management), nutzen wir im Rahmen dieser Arbeit lediglich die folgende: Alle notwendigen Tätigkeiten, die zur Umsetzung einer Lieferkette benötigt werden.

\subsection{Internetrecherche}

Ergänzt wurde die Literaturrecherche durch eine Internetrecherche, weil nicht alle Ontologien durch wissenschaftliche Literatur dokumentiert sind.
Zur Internetrecherche wurden die folgenden Webseiten genutzt:

\begin{enumerate}
    \item IndustryPortal (von OntoCommons) (\url{https://industryportal.enit.fr/}) - TODO Ontologie-Bibliothek ... erwähnen, dass es Anfang März offline ging.
    \item TODO (\url{https://data.ontocommons.linkeddata.es/index}) - TODO Ontologie-Bibliothek ... Was? Warum?
    \item Github (\url{https://www.github.com}) - TODO Was? Warum?
       \begin{itemize}
           \item https://github.com/search?q=topic%3Aontology+industry&type=repositories&ref=advsearch&p=1
           - Topic: ontology
           - Suchbegriffe: industry, manufacturing
       \end{itemize}
    \item Projektseite der Basic Formal Ontology (kurz BFO, \url{https://basic-formal-ontology.org/users.html}) - Enthält eine Liste von Ontologie-Projekten, welche die BFO nutzen.
\end{enumerate}

TODO: Hin- und Her-Bouncing erklären? Ausgangspunkt > schauen was es an Ontologien gibt > Kriterien verfeinern > ...

Weiterhin wurden die Import-Statements in den RDF-Daten der Ontologien manuell geprüft und etwaige Verlinkungen in die Recherche aufgenommen.

\subsection{Selektion einer Ontologie}

Im Rahmen der Literatur- und Internetrecherche wurde eine Ontologie selektiert, wenn sie die folgenden Kriterien erfüllte:

\begin{enumerate}
    \item Die Ontologie besitzt eine URL zu einer Textdatei, welche in einer formalen Sprache (RDF/OWL oder KIF) vorliegt.
    \item Die Ontologie stellt ein Vokabular zur Beschreibung des Fachgebietes bereit.
    \item Der Ontologie liegt eine logische Theorie (z.B. in Form von Axiomen, Regeln, Hierarchien) über das Fachgebiet zugrunde, die auf das Vokabular zurückgreift.
    \item Die Inhalte der Ontologie haben einen direkten thematischen Bezug zu den vorher genannten Schlüsselwörtern bzw. den damit zusammenhängenden Themen.
\end{enumerate}

\textbf{TODO: Schlüsselwort => Thema hier angemessen oder sollte das extra behandelt werden?}

Jede selektierte Ontologie wurde anschließend auf eine Reihe von objektiven Angaben hin untersucht.
Sie helfen dabei, einen neutralen Blick auf die vorliegenden Industrie-Ontologien zu erhalten.
Im Folgenden die Angaben mit einer kurzen Erklärung:

\begin{itemize}
    \item \textbf{Name der Ontologie} - Der Name der Ontologie, der in den RDF-Daten bzw. der zugehörigen Dokumentation zu finden ist.
    \item \textbf{Abkürzung} - Falls vorhanden, die Abkürzung des Namens der Ontologie. Wird häufig als Prefix in den RDF-Daten verwendet.
    \item \textbf{Kurzbeschreibung} - Eine kurze, prägnante Beschreibung über den Inhalt der Ontologie.
    \item \textbf{Autor(en)} - Eine Liste von Namen der Autoren oder beteiligten Gruppen/Unternehmen.
    \item \textbf{Projektseite oder Publikation} - Eine URL auf die Projektseite, falls vorhanden. Alternativ eine URL zur zugehörigen Publikation.
    \item \textbf{Link zu Rohdaten (RDF- oder KIF-Format)}: Eine URL zu einer oder mehreren Dateien mit den Rohdaten im RDF- oder KIF-Format.
    \item \textbf{Neuste Version} - Falls vorhanden, eine Angabe zur neusten Version der Rohdaten.
    \item \textbf{Datum der neusten, dokumentierten Änderung} - Eine Datumsangabe der neusten und dokumentierten Änderung an den Rohdaten.
    \item \textbf{Lizenz} - Falls vorhanden, eine Angabe zu den verwendeten Lizenzen.
\end{itemize}


\section{Rechercheergebnisse}

[...]

\textbf{TODO: binde finale Version der Grafik ontology-map.png hier ein}

\textbf{TODO: Zusammenfassung der Rechercherergebnisse}

Weiterführende Metadaten zu den Ontologien finden Sie in unseren Rohdaten, welche in Form einer CSV-Datei öffentlich zur Verfügung stehen.

\textbf{TODO Link zu }

\subsection{Referenzierte Ontologien}

In diesem Kapitel werden alle Ontologien aufgelistet, die keinen thematischen Bezug zur Industrie haben, jedoch von manchen Ontologien mit Industrie-Bezug verwendet werden.
Die Einträge sind alphabetisch sortiert und unter ihnen befinden sich Top-Level- und Kern-Ontologien.
Jede Ontologie wird im Rahmen einer kurzen Zusammenfassung vorgestellt.

\subsubsection{Basic Formal Ontology (BFO)}

Die Basic Formal Ontology (BFO)\footnote{Projektseite: \url{https://basic-formal-ontology.org/}} ist eine Top-Level Ontologie, welche initial 2002 im Rahmen von BFO Project von Barry Smith und Pierre Grenon entwickelt wurde.
Sie eignet sich theoretisch für den Einsatz in beliebigen Fachgebieten, weil sie nur sehr allgemeine Begriffe (und deren Beziehungen untereinander) zur Beschreibung der Realität definiert.
Die BFO teilt die Realität in zwei Hauptzweige ein: in Kontinuanten (Entitäten, die ohne zeitlichen Bezug existieren) und Okkurrenten (Entitäten, die eine zeitliche Ausdehnung haben).
BFO wird von verschiedenen (wissenschaftlichen) Gemeinschaften verwendet, um domänenspezifische Ontologien zu erstellen und zu harmonisieren, z.B. in der Biomedizin, der Geographie, der Soziologie, etc.
Die Ontologie wird stetig weiterentwickelt und liegt aktuell in Version 2.0 vor\footnote{Quelle: \url{https://github.com/BFO-ontology/BFO?tab=readme-ov-file\#implementations}}.

\subsubsection{GoodRelations Ontologie}

Die GoodRelations-Ontologie\footnote{https://www.heppnetz.de/ontologies/goodrelations/v1.html} istlaut Aussage ihres Autores Martin Hepp ein Vokabular zur Modellierung von Daten im E-Commerce Bereich (z.B. Produkte, Dienstleistungen oder Unternehmen).
Sie wurde mit dem Ziel entwickelt, Produkte und Dienstleistungen auf E-Commerce-Webseiten mit zusätzlichen Metadaten auszuzeichnen, um die Sichtbarkeit in Suchmaschinen zu erhöhen.
GoodRelations wurde 2012 als E-Commerce Kern in das Vokabular schema.org aufgenommen\footnote{\url{http://blog.schema.org/2012/11/good-relations-and-schemaorg.html}}.
Damit gingen alle Klassen und Properties in das schema.org Vokabular über und wurden darin integriert\footnote{Beispiel: Klasse Product \url{https://schema.org/Product}}.
Die Ontologie existiert seit der Aufnahme in Schema.org faktisch nicht mehr und wird hier nur noch erwähnt, weil sie in manchen Ontologie-Publikationen explizit erwähnt wird.

\subsubsection{Quantities, Units, Dimensions and Types ontology (QUDT)}

\textbf{TODO: Text glattziehen, Projektseite und letzte Aktivität}

Unit-Measure-Ontologien stellen eine konzeptuelle Repräsentation von Quantitäten, Quantitätsarten und Einheiten dar.
Der bekannteste und weitläufig eingesetzte Vertreter, ist die von der NASA entwickelte QUDT\cite{QUDTOntology}.
Innerhalb dieser Ontologie werden im Kern die Konzepte physikalischer Quantitäten.
Quantitätsarten, Einheiten und Dimensionen, sowie Datentypen repräsentiert.
Die Ontologie basiert auf dem internationalen Standard der SI-Unit (franz. Système international d'unités) und liegt im OWL-Format vor. Ein Hauptanwendungszeck einer solchen Ontologien, ist das Konvertieren von physikalischen  Einheiten unter Berücksichtigung der Dimensionalität.
QUDT wird eingesetzt in Luft- und Raumfahrt, Industrieproduktion, sowie Energie- und Wasserversorgung.
Ein Beispiel ist die Verwendung von QUDT innerhalb der Systems-Architecture-Ontology (SAO) für Cyber-Physical-Systems.

\subsubsection{schema.org Vocabulary}

Schema.org\footnote{\url{https://www.schema.org}} ist eine von Firmen und Freiwilligen gepflegte quelloffene Ontologie, mit deren Hilfe man Daten auf Webseiten auszeichnen kann.
Das Ziel ist es, Suchmaschinen zusätzliche Metadaten über die Webseiteninhalte bereitstellen zu können.
Auf einer dezidierten Entwickler-Seite\footnote{\url{https://schema.org/docs/developers.html}} findet man u.a. weiterführende Links (z.B. FAQ) und einen Bereich, wo man sich die Rohdaten in verschiedenen RDF-Dialekten herunterladen kann.

\subsubsection{Suggested Upper Merged Ontology (SUMO)}

Die SUMO\footnote{Projektseite: \url{https://www.ontologyportal.org/index.html}} Ontologie ist eine Top-Level Ontologie, deren Entwicklung laut Projektseite von Adam Pease (technical editor) geleitet wird.
Die Ontologie ist in der Sprache "Standard Upper Ontology Knowledge Interchange Format" (SUO-KIF) geschrieben\footnote{\url{https://github.com/ontologyportal/sigmakee/blob/master/suo-kif.pdf}} und bestand anfangs aus Ontologie-Beiträgen auf einer Mailing-Liste, die jedoch mit der Zeit stark erweitert wurden (\cite{niles2001towards}, S. 2 ff.).
SUMO stellt Definitionen für sehr allgemeine Begriffe bereit, welche für abgeleitete Kern-Ontologien verwendet werden können.
Laut eigener Aussage wird sie für die Forschung und Entwicklung von Anwendungen aus den Bereichen Suche, Linguistik und Reasoning verwendet \footnote{Quelle: \url{https://www.ontologyportal.org/index.html}}.
Die KIF-Dateien des Projektes sind thematisch aufgeteilt (z.B. Anatomy.kif für Anatomie, Military.kif für Militärisches) und in einem Github-Repository\footnote{\url{https://github.com/ontologyportal/sumo}} zu finden.
Das Projekt scheint weiterhin gepflegt zu werden, denn der letzte Commit ist vom 4. März 2024\footnote{\url{https://github.com/ontologyportal/sumo/commit/596147ccc5636e20441875a6a29158d8567d7f80}}.

\subsection{Ontologien}

\subsubsection{Additive manufacturing ???}

Paper "A product life cycle ontology for additive manufacturing": https://philarchive.org/archive/ALIAPLv1

Gibt es dazu noch irgendwo die öffentlich zugänglichen RDF Daten?


ODER

Prüfe, ob sie zu folgender Publikation Rohdaten finden lassen:

A Design for Additive Manufacturing Ontology:
\url{https://citeseerx.ist.psu.edu/document?repid=rep1&type=pdf&doi=b18539e972dfd14782d6a907dd04d1b91819e8b2}

\subsubsection{Additive manufacturing ontology (AMontology)}

https://github.com/iassouroko/AMontology

\subsubsection{Automotive Industry Ontology}

Automotive Industry Ontology

https://github.com/iurianu/auto-ontology/tree/master

Frage an Autor ob Projekt noch forgesetzt wird: https://github.com/iurianu/auto-ontology/issues/1

\subsubsection{Battery Interface Ontology (BattINFO)}

https://big-map.github.io/BattINFO/about.html

https://github.com/BIG-MAP/BattINFO?tab=readme-ov-file

https://www.big-map.eu/dissemination/battinfo

Passend?

\subsubsection{Battery Value Chain Ontology (BVCO)}

https://gitlab.cc-asp.fraunhofer.de/ISC-Public/ISC-Digital/ontology/bvco

\subsubsection{Building ontology}

https://bimerr.iot.linkeddata.es/def/building/

\subsubsection{Car Options Ontology}

Die Car Options Ontology (COO)\footnote{Entsprechender Eintrag bei Linked Open Vocabularies: \url{https://lov.linkeddata.es/dataset/lov/vocabs/coo}} stellt ein Vokabular zur Beschreibung von Konfigurationsoptionen in Fahrzeugen bereit.
Dabei können ebenso Abhängigkeiten und Kompatibilitäten zwischen Konfigurationen modelliert werden.
Der Autor der Ontologie ist Martin Hepp.
Sie verwendet Inhalte aus der GoodRelations Ontologie, die nun Teil von Schema.org ist.
Die ursprüngliche Projektseite \url{http://www.volkswagen.co.uk/vocabularies/coo/ns} leitet zu \url{https://www.volkswagen.co.uk/en.html} weiter.
Es konnten keine Anzeichen für eine aktive Betreuung der Ontologie gefunden werden.
Auf Linked Open Vocabularies kann eine nutzbare RDF-Version heruntergeladen werden\footnote{\url{https://lov.linkeddata.es/dataset/lov/vocabs/coo/versions/2010-10-12.n3}}.

\subsubsection{Capability and Skill Ontology (CaSk)}

"OWL ontology to describe capabilities and skills of things. [...] In the context of automated production, Capabilities and Skills are terms that are used to refer to machine functions."

https://github.com/CaSkade-Automation/CaSk

\subsubsection{COGITO Safety Ontology}

https://github.com/oeg-upm/cogito-safety-ontology/tree/main

Prüfe, ob das für uns relevant ist. Projekt fiel mit im Kontext von https://github.com/mahsa-teimourikia/Safety-Ontology auf.

\subsubsection{Collaborative Manufacturing Services Ontology}

https://data.ontocommons.linkeddata.es/vocabulary/CollaborativeManufacturingServicesOntology

PDF: https://www.sciencedirect.com/science/article/pii/S2405896318314472

\subsubsection{Context Aware System Observation Ontology}

https://irstea.github.io/caso/OnToology/ontology/caso.owl/documentation/index-en.html

Verwendet Semantic Sensor Network Ontology

\subsubsection{CORA Ontology}

% https://repositorio.ipcb.pt/bitstream/10400.11/2815/1/proceedings_IROS2014_workshop_IEEE_ORA.pdf#page=13

Die Core Ontology for Robotics and Automation (abgekürzt: CORA) \cite{prestes2014core} wurde von der IEEE RAS Ontology for Robotics and Automation Working Group (IEEE RAS ORA WG) entwickelt und ist ein Ansatz zur Standardisierung der vorhandenen Robotik-Terminologie.
Diese Entwicklung war Teil einer Initiative zur Schärfung der Terminologien u.a. in den folgenden Themengebieten: Industrieroboter (Industrial Robotics), Roboter-assistierte Chirugie (Surgical Robotics) and Service Roboter (Service Robotics).
Mit CORA soll definiert werden, was ein Roboter ist, daher werden u.a. die Konzepte Roboter (robot), Robotisches System (robotic system) und Bestandteil eines Roboters (robotic part) näher beschrieben.
Im Rahmen der Initiative wurden weitere Ontologien entwickelt, die auf CORA aufbauen.
CORA nimmt dabei eine Vermittlerrolle ein, sodass darauf aufbauende Ontologien konsistente Konzepte verwenden können, was zu langfristiger Kompatibilität beitragen kann.
Die Ontologie baut dabei der auf der Top-Level Ontologie SUMO auf, wobei weitere Unterontologien von CORA entwickelt wurden, um semantische Lücken zwischen SUMO und CORA zu überbrücken (siehe Unterkapitel für CORAX, POS und RPARTS).

Auf der Projektseite \url{https://github.com/srfiorini/IEEE1872-owl} findet man die OWL Spezifikation von CORA und anderen Ontologien hinter dem IEEE 1872-2015 Standard\footnote{\url{https://standards.ieee.org/ieee/1872/5354/}}. Der letzte Commit im Repository ist vom 21.10.2020\footnote{\url{https://github.com/srfiorini/IEEE1872-owl/commit/ffb6613350b28c3a710f8daf2a1e9e953ded7ed8}}, weshalb man annehmen kann, dass das Projekt nicht mehr aktiv betreut wird.

\subsubsection{CORAX Ontology}

% PDF: https://www.sciencedirect.com/science/article/am/pii/S0736584514000659

Die CORAX Ontologie wurde im Rahmen der CORA Ontologie entwickelt, mit dem Ziel notwendige Konzepte zu definieren, die für die CORA zu allgemein waren, aber in der Top-Level Ontologie SUMO nicht existierten (\cite{fiorini2015extensions}, S. 3).
Dazu gehören Definitionen zu den Konzepten Design (design, im Sinne von Produktdesign), physische Umgebung (physikal environment), Interaktion (interaction) und Künstliches System (artifical system).
So wurde zum Beispiel zwischen SUMO:Artificat und CORA:Robotic System das Konzept CORAX:Artificial System (CORAX) eingeführt (\cite{fiorini2015extensions}, S. 6 ff.).

Auf der Projektseite \url{https://github.com/srfiorini/IEEE1872-owl} findet man neben der OWL Spezifikation von CORA auch die von CORAX. Der letzte Commit im Repository ist vom 21.10.2020, weshalb man annehmen kann, dass das Projekt nicht mehr aktiv betreut wird.

\subsubsection{Crystallography Domain Ontology}

https://github.com/emmo-repo/domain-crystallography

\subsubsection{Defection Ontology (DefectOnt)}

\url{https://www.researchgate.net/publication/364530871_An_Ontology_for_Defect_Detection_in_Metal_Additive_Manufacturing}

https://github.com/AndreaMazzullo/DefectOnt (kleine OWL Datei, aber womöglich brauchbar und erwähnenswert)

\subsubsection{Digital Construction Energy Ontology}

https://digitalconstruction.github.io/Energy/v/0.5/

\subsubsection{Digital Construction Entities Ontology}

https://digitalconstruction.github.io/Entities/v/0.5/

\subsubsection{Digital Construction Materials}

https://digitalconstruction.github.io/Materials/v/0.5/

\subsubsection{Domain ontology for atomistic and electronic modelling}

https://github.com/emmo-repo/domain-atomistic

\subsubsection{Elementary Multiperspective Material Ontology (EMMO)}

https://github.com/emmo-repo/EMMO

\subsubsection{Energy Efficiency Prediction Semantic Assistant Ontology (EEPSA)}

https://iesnaola.github.io/eepsa/EEPSA/index-en.html

Verwendet Semantic Sensor Network Ontology

\subsubsection{European Materials \& Modelling Ontology (EMMO)}

TODO: https://github.com/emmo-repo/ + https://emmc.eu/news/emmo-1-0-0-alpha-release/

\subsubsection{Extruder Ontology (ExtruOnt)}

https://data.ontocommons.linkeddata.es/vocabulary/Extruont

zugehöriges Paper: https://www.semantic-web-journal.net/system/files/swj2317.pdf

\subsubsection{General Process Ontology (GPO)}

https://gitlab.cc-asp.fraunhofer.de/ISC-Public/ISC-Digital/ontology/gpo

Ontologie ist noch WIP

\subsubsection{ifcOWL}

"The IFC4\_ADD1 ontology specifies in OWL the Industry Foundational Classes (IFC) conceptual data schema and exchange file format for Building Information Model (BIM) data; Creators: Pieter Pauwels (pipauwel.pauwels@ugent.be) and Walter Terkaj (walter.terkaj@itia.cnr.it); Reference: http://www.w3.org/community/lbd/ifcowl/; Based on IFC schema http://www.buildingsmart-tech.org/ifc/IFC4/final/html/"

\url{https://ontohub.org/spaceportal/IFC%20OWL/IFC4_ADD1///symbols?kind=ObjectProperty}

\url{https://standards.buildingsmart.org/IFC/DEV/IFC4/ADD2_TC1/OWL/index.html}

https://technical.buildingsmart.org/standards/ifc/ifc-schema-specifications/

https://github.com/buildingSMART/ifcOWL

Prüfen, ob das noch für uns passt, weil inhaltlicher Fokus eher Bauwirtschaft ist

\subsubsection{Industrial Data Ontology}

https://rds.posccaesar.org/ontology/lis14/ont/core/

Die Verknüpfung mit dem Standard ISO 15926-2:2003 macht sie interessant

\subsubsection{Industrial Ontologies Foundry Core Ontology}

IOF = The Industrial Ontologies Foundry

TODO: \cite{kulvatunyou2022}

\subsubsection{Industrial Ontologies Foundry Supply Chain Reference Ontology}

Die Supply-Chain-Reference-Ontology (IOF-SCRO) wurde als eine Referenz-Ontologie für die Domänen Lieferketten-Management (Supply Chain Management) und Logistik entwickelt\cite{ameri2020towards}.
Die Autoren sind Teil der Industrial Ontologies Foundry (IOF) Supply Chain Working Group unter Leitung von Farhad Ameri\footnote{\url{https://oagi.org/pages/supply-chain-working-group}}.
Die primären Anwendungsfälle der Ontologie sind Lieferkettenverfolgung (Traceability Use-Case) und Lieferantenermittlung (Supplier-Discovery), wofür sie entsprechende Inhalte bereitstellt.
Sie wurde mit dem Ziel entwickelt, eine ontologische Basis bereitzustellen, um die Konsistenz und Interoperabilität zwischen Ontologien aus der Logistik und dem Lieferketten-Management zu unterstützen.
Man nutzt die BFO als Top-Level-Ontologie genutzt und verwendet weiterhin Teile der IOF Core Ontology\cite{Kulvatunyou2022}.
Die Projektseite\footnote{\url{https://spec.industrialontologies.org/iof/ontology/supplychain/SupplyChainReferenceOntology/}} der Ontologie beinhaltet weitere Referenzen und Angaben zum Projekt (z.B. Lizenz, Autoren).
Die Ontologie ist auch auf anderen Portalen zu finden, z.B. dem IndustryPortal\footnote{\url{https://industryportal.enit.fr/ontologies/IOF-SCRO}}.
Das Projekt scheint noch aktiv gepflegt zu werden, denn die neuste Veröffentlichung der Ontologie ist vom 01.01.2023\footnote{Anhand von \url{https://industryportal.enit.fr/ontologies/IOF-SCRO}, Bereich Submissions}.

\subsubsection{isa\_tab\_ontology}

Wird bei OntoCommons genannt, es ist jedoch noch nicht ersichtlich, ob das für uns passt.
Es geht hierbei scheinbar um ISA-Tab:
- https://www.dcc.ac.uk/resources/metadata-standards/isa-tab
- https://isa-tools.org/index.html

https://data.ontocommons.linkeddata.es/vocabulary/Isa\_tab\_ontology

https://github.com/2kunal6/OntologyWebform/tree/master


\subsubsection{Key Performance Indicator ontology}

https://bimerr.iot.linkeddata.es/def/key-performance-indicator/

\subsubsection{Maintenance Activity Ontology}

https://www.semantic-web-journal.net/system/files/swj3067.pdf

\url{https://github.com/uwasystemhealth/Paper_Archive_Maintenance_Activity}

\subsubsection{Manufacturing Failure Prediction Ontology (MFPO)}

https://data.ontocommons.linkeddata.es/vocabulary/ManufacturingFailurePredictionOntology(mfpo)

https://www.sciencedirect.com/science/article/pii/S1877050919314024/pdf?md5=221db728bbcefb3d067b0b8cabac8cec\&pid=1-s2.0-S1877050919314024-main.pdf

RDF-Datei vorhanden?

\subsubsection{Mappings Ontology}

https://github.com/emmo-repo/domain-mappings

\subsubsection{Material properties ontology}

https://bimerr.iot.linkeddata.es/def/material-properties/

\subsubsection{Material Science and Engineering Ontology (MSEO)}

"MSEO utilizes the Common Core Ontology stack giving materials scientists and engineers the ability to represent their experiments and resulting data. The goal is to create machine and human readable sematic data which can be easily digested by other science domains. It is a product of the joint venture Materials Open Lab Project between the Bundesanstalt für Materialforschung und -prüfung (BAM) and the Fraunhofer Group MATERIALS and used the BWMD ontology created by Fraunhofer IWM as a starting point."

https://matportal.org/ontologies/MSEO

\subsubsection{Materials And Molecules Basic Ontology (MAMBO)}

Paper: https://arxiv.org/pdf/2111.02482.pdf

https://github.com/daimoners/MAMBO

\subsubsection{Materials Design Ontology}

https://github.com/LiUSemWeb/Materials-Design-Ontology

Ist nicht von OntoCommons, aber wurde in einer ihrer verlinkten Ontologies erwähnt.
Könnte auch bei uns reinpassen, weil die abgedrehte Materialwissenschaften machen.

könnte zugehöriges Paper sein: https://arxiv.org/abs/2006.07712

\subsubsection{Mechanical Testing Ontology}

https://github.com/emmo-repo/domain-mechanical-testing

\subsubsection{Microstructure Domain Ontology}

https://github.com/emmo-repo/domain-microstructure

\subsubsection{MPFQ Ontology (Material-Process-Function-Quality)}

https://data.ontocommons.linkeddata.es/vocabulary/MpfqOntology(material-process-function-quality)

Könnte passen, jedoch hab ich auf die schnelle keine Literatur/Doc dazu gefunden.

\subsubsection{Ontology for Autonomous Robotics (ROA / ORA)}

TODO lesen: https://www.aolivaresalarcos.com/pdf/2017-roman.pdf
cite \cite{olszewska2017ontology}

\subsubsection{Ontology for Industry 4.0 (O4I4)}

\cite{kumar2019ontologies}

Grafik machen: O4I4 nutzt CORA, ROA/ORA und ORArch (\cite{kumar2019ontologies}, S. 7)

\subsubsection{Ontology for manufacturing and logistics (OFM)}

https://data.ontocommons.linkeddata.es/vocabulary/ManufacturingSystemOntology

Wird bei OntoCommons anders genannt als im Github-Repository

https://github.com/enegri/OFM/blob/master/README.pdf

\subsubsection{Ontology for Reliability-centred MAintenance (ORMA)}

Ontology-augmented Prognostics and Health Management for shopfloor-synchronised joint
maintenance and production management decisions

https://re.public.polimi.it/bitstream/11311/1193315/6/IRIS-ORMA-post-print.pdf

\textbf{TODO}

\subsubsection{Ontology for Robotic Architecture (ORArch)}

\cite{kumar2019ontologies}

\subsubsection{Ontology of units of Measure (OM)}

Die Ontology of units of Measure (OM) wurde von dem Wissenschaftler Hajo Rijgersberg im Rahmen seiner Tätigkeit in der Abteilung Computer-Science der Universität Wageningen (Niederlande) entwickelt, als integrativer Teil einer größeren Ontologie\cite{rijgersberg2013ontology}, der Ontology of Quantitative Research (OQR).
Die OM Ontologie versucht die Konzepte der Domäne von Maßen und Einheiten umfassend abzubilden, welche in Teilen bei anderen Ontologien fehlen.
Dies betrifft z.B. die Konzepte Quantity und Prefix.
Die OM Ontologie bezieht sich dabei in Teilen auf die QUDT Ontologie und kommt in verschiedenen Ontologie-Projekten, wie z.B dem Bioportal\cite{OM_BioPortal}.

Auf der Projektseite \url{https://github.com/HajoRijgersberg/OM} steht die Ontologie in der Version 2.0 zur Verfügung.
Die Ontologie wird aktiv betreut, denn der letzte Commit im Repository is vom 08.02.2024.

\subsubsection{Ontology-based Customization and Visualization of Information Flow control in an Industry 4.0 scenario ????}

Könnte relevant sein, weil auf der README ein paar Ontologien im Kontext von Industrie 4.0 erwähnt werden: \url{https://github.com/ko3n1g/Ontology-based-InformationFlow-Industry-4.0?tab=readme-ov-file#ontologies}

\subsubsection{Open Energy Ontology}

Achtung, enthält mehrere TeilOntologien

https://openenergy-platform.org/ontology/oeo/

https://github.com/OpenEnergyPlatform/ontology

Grund für mögliche Relevanz: Im OEO-Physical Teil geht es u.a. um Kraftwerke, Generatoren etc. welche im Rahmen der Industrie eine wichtige Rolle spielen.

\subsubsection{ORMA+ Ontology}

"Zero Defect Manufacturing in the Digital Twin Framework"

\url{https://www.politesi.polimi.it/bitstream/10589/195722/3/2022_10_Ghedini.pdf}

Prüfe, ob es hierfür öffentlich zugängliche RDF Daten gibt. Im Paper konnte ich auf die schnelle nichts finden, jedoch zeigen sie darin Protégé Screenshots

\subsubsection{Position Ontology (POS)}

% PDF: https://www.sciencedirect.com/science/article/am/pii/S0736584514000659

Die Position Ontology (POS) wurde als Erweiterung der SUMO Ontologie im Rahmen des CORA Projektes entwickelt (\cite{fiorini2015extensions}, S. 17 ff).
Sie komplementiert die CORA Ontologie mit neuen Konzepten zur Positionierung und Orientierung, was insbesondere bei der Modellierung von Informationen über Roboter und ihre Umgebung wichtig ist.
Dafür wurden u.a. die folgenden Konzepte eingeführt: position, PositionCoordinateSystem, PositionPoint.
Diese und andere Konzepte und Properties erlauben die Modellierung von Positionen in einem Koordinatensystem sowie die relative Positionierung.

Auf der Projektseite \url{https://github.com/srfiorini/IEEE1872-owl} findet man neben der OWL Spezifikation von CORA auch die von POS. Der letzte Commit im Repository ist vom 21.10.2020, weshalb man annehmen kann, dass das Projekt nicht mehr aktiv betreut wird.

\subsubsection{Product Lifecycle Management Ontology???}

https://rds.posccaesar.org/ontology/plm/

Prüfe, ob hierfür noch passende Dokumentation vorhanden ist. Es scheint auf jeden Fall RDF-Daten vorhanden zu sein

\subsubsection{Production Equipment ontology ???}

"Core Ontology for Describing Production Equipment According to Intelligent Production":
https://www.mdpi.com/2571-5577/5/5/98

Prüfe, ob es dafür öffentlich zugängliche RDF Daten gibt

\subsubsection{Product Ontology}

http://www.productontology.org/ (nutzt GoodRElations)

[...]

\subsubsection{Product Ontology (PRONTO)}

https://ri.conicet.gov.ar/bitstream/handle/11336/13365/CONICET\_Digital\_Nro.16500.pdf?sequence=1

https://data.ontocommons.linkeddata.es/vocabulary/ProductOntology(pronto)

Konnte keine RDF-Dateien finden, hier nochmal schauen

\subsubsection{RealEstateCore Ontology (REC)}

https://www.diva-portal.org/smash/get/diva2:1362356/FULLTEXT01.pdf

https://doc.realestatecore.io/3.0/full.html\#

Aufgenommen aufgrund des Papers, darin werden die folgenden Schlüsselwörter verwendet:  ontology, smart buildings, building automation, IoT, energy optimization

Prüfe, ob es noch in den Scope passt

\subsubsection{Reference Generalized Ontological Model (RGOM)}

Paper: https://www.mdpi.com/2076-3417/11/11/5110/pdf

https://github.com/MuhammadYahta/rgom/tree/main

\subsubsection{Reified Requirements Ontology}

https://data.ontocommons.linkeddata.es/vocabulary/ReifiedRequirementsOntology

https://data.dnv.com/ontology/requirement-ontology/documentation/req-ont.pdf

Konnte keine RDF-Dateien finden, hier nochmal schauen

\subsubsection{RFID System Configuration Ontology (RDFID SCO)}

"Enhancing RFID system configuration through semantic modelling": \url{https://repository.lboro.ac.uk/articles/journal_contribution/Enhancing_RFID_system_configuration_through_semantic_modelling/14805819/1/files/28480563.pdf}

\url{https://github.com/eleniTsalapati/ONTOLOGIES/blob/master/RFID_SCO.owl}

\subsubsection{RPARTS Ontology}

% PDF: https://www.sciencedirect.com/science/article/am/pii/S0736584514000659

Die RPARTS Ontology ist eine Teil-Ontologie der CORA Ontology, welche zur Modellierung von Bestandteilen von Robotern verwednet werden kann. (\cite{fiorini2015extensions}, S. 14 ff).
Der zugrundeliegende Ansatz ist, dass Roboter (\textit{devices}) immer aus Bestandteilen (anderen \textit{devices}) bestehen.
Man unterscheidet zwischen Roboter-Bestandteilen zur Kommunikation (\textit{Robot communicating part}), Verarbeitung (\textit{Robot processing part}), Sensorik (\textit{Robot sensing part}) und Handlung (\textit{Robot actuating part}).
[...]?

Auf der Projektseite \url{https://github.com/srfiorini/IEEE1872-owl} findet man neben der OWL Spezifikation von CORA auch die von POS. Der letzte Commit im Repository ist vom 21.10.2020, weshalb man annehmen kann, dass das Projekt nicht mehr aktiv betreut wird.

\subsubsection{Safety Ontology}

"This is an ontology for extracting the safety knowledge in industry 4.0."

https://github.com/mahsa-teimourikia/Safety-Ontology

\subsubsection{Semantic Sensor Network (SSN) ontology}

https://w3.org/TR/vocab-ssn/

Im Kontext mit SOSA

\subsubsection{Sensor Data ontology}

https://bimerr.iot.linkeddata.es/def/sensor-data/

\subsubsection{Sensor, Observation, Sample, and Actuator Ontology (SOSA)}

https://w3.org/TR/vocab-ssn/

Semantic CPPS in Industry 4.0 >> https://arxiv.org/pdf/2011.11395.pdf

Im Kontext mit SSN

\subsubsection{Smart Applications Reference Ontology (SAREF)}

https://saref.etsi.org/

\subsubsection{Smart Applications Reference Ontology extension for building (SAREF4BLDG)}

https://saref.etsi.org/saref4bldg

Ist eine Erweiterung der SAREF Ontology.

\subsubsection{Smart Applications Reference Ontology extension for industry and manufacturing (SAREF4INMA)}

https://saref.etsi.org/saref4inma

Ist eine Erweiterung der SAREF Ontology.

\subsubsection{Vehicle Sales Ontology (VSO)}

Quelle: https://www.heppnetz.de/ontologies/vso/ns

Wird genutzt von VVO (IndustryPortal)

nutzt GoodRelations Vokabular

\subsubsection{Versioning Ontology (VERONTO)}

https://ontohub.org/repositories/veronto

https://ri.conicet.gov.ar/bitstream/handle/11336/6899/CONICET\_Digital\_Nro.9341\_A.pdf?sequence=2

\subsubsection{VICINITY project}

https://vicinity2020.eu/vicinity/node/446

https://github.com/mariapoveda/vicinity-ontology-core

Projekt könnte von Interesse sein, weil: "Open virtual neighbourhood network to connect IoT infrastructures and smart objects"

\subsubsection{Volkswagen Vehicles Ontology (VVO)}

Aufgrund fehlender Originalquelle (z.B. wiss. Publikation oder Projektseite) und der unterschiedlichen Dokumentation zur Volkswagen Vehicles Ontology, werden die aktuellen Funde im Folgenden einzeln dargelegt.

Die Ontologie wird in einer Untersuchung namens "Case Study: Contextual Search for Volkswagen and the Automotive Industry" des W3C aus dem Jahre 2011 erwähnt (\cite{greenly2011case}, S. 2).
In der Studie ging es u.a. um die Verbesserung der Suche für Nutzer in digitalen Quellen.
In diesem Zuge wurde u.a. die Volkswagen Vehicles Ontology entwickelt, um Volkswagen-spezifische Konzepte bei Fahrzeugen beschreiben zu können.
Der Autor der Ontologie war Martin Hepp.
Man nutzte diese und andere Vokabulare/Ontologien, um die eigenen Daten anzureichern, um kontextbasierte Suchanfragen zu ermöglichen.
Die URL \url{http://purl.org/vvo/ns} wird als Quelle für die Ontologie angegeben, jedoch gelangt man darüber am Ende nur zu \url{https://www.volkswagen.co.uk/en.html}, welches keine weiteren Informationen bereitstellt.
Weitere Details zur Ontologie selbst wurden nicht genannt.

Der Link \url{https://industryportal.enit.fr/ontologies/VVO} führt zu einem Projekteintrag auf IndustryPortal.
Als Autor wird Abdelouadoud Rasmi angegeben, der jedoch keine ersichtliche Verbindung zu Volkswagen hat.
In der zugehörigen OWL-Datei\footnote{https://data.industryportal.enit.fr/ontologies/VVO/submissions/1/download?apikey=019adb70-1d64-41b7-8f6e-8f7e5eb54942} findet man jedoch im Abschnitt Autor auch Martin Hepp und bei Beitragende (contributors) u.a. den Erstautor der oben genannten W3C Studie William Greenly.
Aufgrund dieses und anderer Gemeinsamkeiten, z.B. gleicher Name (Volkswagen Vehicles Ontology) und XML-Namensraum (http://purl.org/vvo/ns\#), wird angenommen, dass es sich um die gleiche Ontologie handelt. Die Ontologie auf IndustryPortal hat als Veröffentlichungsdatum den 2. November 2022, wobei als Veröffentlichungsdatum bei der Version der 12. Oktober 2010 genannt wird.
Es liegen keine Informationen vor, dass zwischen diesen Daten weitere Veränderungen an der Ontologie vorgenommen wurden.
Ich gehe daher davon aus, dass es sich hier lediglich um einen erneuten Upload der Version von 2010 handelt.

Für die Ontologie existiert auch auf DataScientia ein Eintrag unter \url{http://liveschema.eu/dataset/lov\_vvo}.
Als Autor wird wieder Martin Hepp angegeben.
Der Eintrag wurde jedoch am 3. Februar 2020 erstellt und repräsentiert auch die Version 1.0 vom 12. Oktober 2010.


\section{Fazit und weitere Arbeiten}

[...]

\subsection{Beobachtungen}

- Publikationen, welche jedoch auf nicht mehr erreichbare Projektseiten/RDF-Daten verlinken (link rot)


\textbf{TODO: Erwähnung KI-Werk Projekt}



\medskip

\printbibliography

\end{document}
