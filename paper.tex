\documentclass{article}
\usepackage[utf8]{inputenc}
\usepackage[english]{babel}
\usepackage{caption}
\usepackage{float}
\usepackage{subcaption}
\usepackage[table,xcdraw]{xcolor}

\usepackage[lmargin = {2.7cm},
rmargin = {2.7cm},
tmargin = {2.2cm},
bmargin = {2.2cm}
]{geometry}

\usepackage{comment}
\usepackage[hidelinks]{hyperref}
\usepackage[capitalise]{cleveref}

\usepackage[
backend=biber,
style=numeric,
sorting=nty
]{biblatex}
\addbibresource{paper.bib}

\usepackage{graphicx}
\graphicspath{ {./} }

\title{Bestandsaufnahme zu Ontologien aus dem Bereich der Industrieproduktion}
\author{\textbf{Konrad Abicht} \\ k.abicht@gmail.com}
\date{TODO}

\begin{document}

\maketitle

\begin{abstract}
    TODO
\end{abstract}

% ----------------------

\textbf{Grobe Liste aller Inhalte:}

- Welche Arten von Ontologien? Textuell beschriebene (z.B. GFO-Time), in OWL vorliegend

- Vorstellung eventueller Top Level Ontologien, die genutzt wurden (BFO, DOLCE, ...)

- Herausarbeitung eventueller "Zwischen-"Kernontologien

- Bewertung/Einordnung der Ontologien anhand der folgenden Kriterien:

--+ Thematischer Bezug (Materialwissenschaften, Produktion, ...)

--+ Vorliegende Formate (Textuell z.B. als Paper, OWL/RDF oder KIF)

--+ Umfang anhand der Anzahl von Klassen/Properties etc.

--+ Lizenz / Nutzungsbedingungen

--+ Genutzte Sprache(n) (DE, EN,...)

--+ Wartungszustand? Letzte Änderung/Aktivität des Entwicklers in den letzten x Jahren

--+ Top-Level-Ontologie genutzt?

- Ausarbeitung sollte Rohdaten für spätere Weiterverarbeitung enthalten (z.B. eine CSV-Datei, welche je Zeile Metadaten über eine Ontologie beinhaltet)

- Grafik erstellen, die die Bereich optisch darstellt, die bereits ontologisch abgedeckt sind und die, welche nicht


% ----------------------

\section{Einleitung}

Motivation: Es soll eine Bestandsaufnahme aller Ontologien/Vokabluaren?? gemacht werden, welche sich thematisch auf den Fachbereich der Industrieproduktion beziehen (ohne Agrarproduktion). Dabei liegt der Fokus darin, vorhandene Ontologien, die in einer formalen Sprache vorliegen, thematisch zu ordnen (Überlappungen, Abgrenzungen, Eigenständigkeit) und nach objektiven Kriterien aufzulisten. Der Wert dieser Arbeit für die Ontologie-Community besteht in einer fundierten, aufbereiteten Übersicht von Ontologien zum Thema Industrie. Darauf aufbauend können weitergehende Untersuchungen angeknüpft werden.

[Einleitungstext]

Nur Auswahl von Ontologien, deren zugehörige Publikation(en) bzw. die OWL- oder KIF-Dateien für die Öffentlichkeit zugänglich sind.
Eine Prüfung der vorliegenden Lizenz wurde nicht vorgenommen.


Warum sind Ontologien relevant für heutige Industrie (bzw. Industrie 4.0)?
- https://www.cambridge.org/core/services/aop-cambridge-core/content/view/BF86BB5310356D642C82470D67974804/S0269888919000109a.pdf/div-class-title-ontologies-for-industry-4-0-div.pdf, S. 4 => im Bereich der Cyber-Physical-System Integration stoßen heterogene Daten aufeinander, die integriert werden müssen. Weiterhin muss Integration auch in dem Agentenbasierten-Ökosystem umgesetzt werden, um eindeutige Kommunikation, effiziente Kollaboration und Kooperation zu ermöglichen/gewährleisten
- siehe zweiter Absatz S. 4: "Indeed, ontologies aim
 to make domain knowledge explicit and remove ambiguities, enable machines to reason, and facilitate
 knowledge sharing between machines and humans (Persson + Wallin, 2017) and in between machines
 (Olszewska + Allison, 2018). Moreover, ontologies for the I4.0 are required to be business focused, that
 is promoting cooperation with customers and partners (Persson + Wallin, 2017) and, on the other hand,
 meet ontological, autonomous robotic requirements (Bayat et al., 2016). Furthermore, ontologies need to
 analyze and reuse domain knowledge by using present ontologies (Persson + Wallin, 2017)."

- S. 5: über Smart MAnufactoring => Cheng 2016: erstellte/stellte vor ein Modell für Produktionslinien, bestehend aus 5 Ontologien: (1) device ontology, process ontology, parameter ontology, product ontology und base ontology (für Integration der anderen + dem Konzept Order)
- S. 5: Engel 2018 stellte dagegen eine 3-lagige Ontologie für Fabriken mit Batchverarbeitung (umfangreiche repetive Arbeit) => 1. Layer ist Application Layer, 2. Domain Layer ist die Architektur und 3. Layer beschreibt ein allgemeines ontologisches Modell für die Integration dieser und ggf. weiterer Konzepte

- S. 5: es gibt noch keine Standardisierung bei den Ontologien => weil bei I4.0 Roboter und deren Integration in viele Arbeitsabläufe eine zentrale Rolle spielen, muss deren Integration in die verschiedenen Arbeitsumgebungen sowie die Kommunikation untereinander und mit Menschen (Operatoren, Kunden, ...) so schnell wie möglich vereinheitlich werden ==> IEEE 1872-2015 Standard Ontologies for Robotics and Automation (IEEE-SA) prüfen

This paper is structured as follows: [TODO]


\section{Wichtige Begriffe}

\subsection{Themengebiet und Disziplin}

Im Rahmen dieser Arbeit werden thematisch passende Ontologien untersucht, weshalb im Folgenden die Sichtweise der Formalen Ontologien gewählt wurde. Dafür beziehe ich mich auf die theoretischen Grundlagen hinter der "General Formal Ontology" (kurz GFO). Sie geht davon aus, dass unsere Realität sich in drei ontologische Ebenen aufteilen lässt (auch Strata genannt) \cite{herre2006general}. Diese sind: die materielle Ebene (Raum-Zeit), die mentale/psychologische Ebene und die soziale Ebene. Ein Themengebiet ist ein Ausschnitt aus mindestens einer dieser ontologischen Ebenen.
In der Fachliteratur werden Themengebiete häufig im Zusammenhang mit wissenschaftlichen Disziplinen organisiert. Eine wissenschaftliche Disziplin (auch Fachgebiet oder Einzelwissenschaft) ist eine Menge (mehr oder weniger) abgegrenzter Themengebiete, die sich in Lehre, Forschung und Praxis spezialisieren\footnote{https://de.wikipedia.org/wiki/Fachgebiet}. Die Grenze zwischen den Themengebieten ist fließend und nicht immer eindeutig.


\subsection{Vokabular bzw. (Controlled) Vocabulary}

Unsere Rechercheergebnisse zeigen, dass die Vokabular- und Ontologie-Autoren häufig ihre eigenen Kreationen entweder in eine oder beide Kategorien einordnen.
Aufgrund der Rechercherergebnisse definieren und Praktikabilitätsgründen folgen wir

TODO arbeite \cite{neuhaus2018ontology} in diese und das folgende Unterkapitel ein


\subsection{Ontologien und ihre Arten}

Eine Ontologie im Sinne der Formalen Ontologien ist vordergründig eine strukturierte Form von Konzepten, Kategorien, Fakten, logischen Axiomen und Regeln aus einem Themengebiet, welche in Textform vorliegt. Das kann entweder in Form einer Publikation sein, in der diese Dinge textuell und behandelt, grafisch dargestellt und miteinander verknüpft werden. Axiome und logische Aussagen liegen dabei als formale Sprache vor, häufig Beschreibungslogik oder Prädikatenlogik der 1. Stufe. Eine andere Form sind RDFS-/OWL- bzw. KIF-Dateien, welche die genannten Inhalte entsprechend kodiert beinhaltet.

Im Folgenden werden die Ontologien in eine von drei Gruppen eingeteilt (nach \cite{hoehndorf2009developing}). Die erste Gruppe sind die Top-Level-Ontologien (oder auch "foundational ontologies" genannt). Sie bilden das theoretische Fundament zur Modellierung von Phänomenen aus den genannten ontologischen Ebenen. Zu den bekanntesten Top Level Ontologien gehören DOLCE, SUMO, BFO und GFO.

Die Kernontologien bilden die zweite Gruppe. Sie bauen in der Regel auf einer Top-Level-Ontologie auf und schaffen die Grundlagen für die Modellierung eines Themengebiets. Dazu gehört u.a. das Beschreiben relevanter, aber oftmals noch sehr allgemeiner Konzepte und deren Beziehungen/Verknüpfungen unterinander. Im Vorgrund steht das Abstecken thematischer Bereiche.

Die dritte und letzte Gruppe sind die sogenannten Taskontologien, welche für die Bearbeitung eines konkreten Anwendungsfalls entwickelt und genutzt werden. In der Praxis wird diese Einteilung nicht immer in der Form umgesetzt, jedoch bietet sie sich für die Einordnung vorhandener Ontologien an.

\subsection{Ontology Design Pattern (ODP)}

% PDF: https://ceur-ws.org/Vol-1188/paper_11.pdf

Ein Ontology Design Pattern ist ein Ansatz bei der Ontologie-Entwicklung und wird eingesetzt, wenn wiederkehrende Modellierungsaufgaben zu lösen sind.
Sie können dabei helfen, durch Wiederverwendung Zeit und Aufwand zu sparen.
Für einen guten Überblick und Erläuterung der Terminologie wird auf die Publikation \cite{falbo2013ontology} verwiesen.

[...]

\section{Thematische Einordnung}

Im vorherigen Kapitel wurden die wichtigsten Begriffe vorgestellt und eingeordnet.
Nun soll der thematische Fokus dieser Publikation herausgearbeitet werden, um darauf basierend entscheiden zu können, welche Ontologien für die Untersuchung in Frage kommen.

Wir folgen den Autoren von \cite{lasi2014industrie}, die die Industrie als Teil der Wirtschaft bezeichnen, der materielle Güter mit einem hohen Grad an Mechanisierung und Automatisierung erstellt.
Dabei spielen angrenzende Themengebiete wie z.B. die Rohstoffförderung, Logistik und digitale Systeme auch eine wichtige Rolle.
Der thematische Fokus dieser Arbeit ist die Industrie und speziell die Güterfertigung.
Angrenzende Themengebiete (z.B. Materialwirtschaft) werden nur soweit notwendig mit betrachtet.

In der aktuellen Literatur (wissenschaftliche Publikationen, White-Paper, Blogs etc.) im Bereich der Industrie scheinen Begriffe aus dem Themengebiet Industrie 4.0 sehr dominant.
Nichtsdestotrotz zeigt die Praxis, dass Industrieunternehmen unterschiedliche Ausprägungsgrade bei der Adaption neuer Entwicklungen im Bereich Produktion haben.
Wir gehen daher davon aus, dass eine breit aufgestellte Betrachtung hier zielführender ist, als sich nur auf den Teil mit Industrie 4.0 zu fokussieren.
Aufgrund der Vielfalt an Begriffen, Buzzwords, Synonymen und Homonymen im Bereich der Industrie wird im Folgenden eine grobe Einordnung der wichtigsten Begriffe vorgenommen.

\subsection{Industrie 4.0}

Der Begriff "Industrie 4.0" wurde 2011 von der deutschen Bundesregierung im Rahmen der Hightech-Strategie 2020 geprägt (\cite{lasi2014industrie} und \cite{trotta2018industry}, S. 1). Eine Verbreitung außerhalb des deutschen Sprachraums fand erst später statt. Ich teile die Ansicht von (\cite{ragavan2016engineering}, S. 1), dass der Begriff \textit{Industrie 4.0} eine Sammlung von vielen Buzzwords und Technologien ist. Allen Definitionen ist weitestgehend gemein, dass es bei der Industrie 4.0 um intelligente Fabriken (smart factory) geht, welche intelligente Produkte (smart products) mittels intelligenter Maschinen (smart machines) herstellen. Der Begriff spielt in vielen Ontologien eine wichtige Rolle, weshalb es zunächst eine grobe Einordnung des Begriffs erarbeitet werden soll. Danach folgt eine Auflistung der zugerechneten Themengebiete.

In der Fachwelt gibt es die verbreitete Ansicht, dass sich die Industrie in den letzten Jahrhunderten stufenweise weiterentwickelt hat. Vereinfacht gesagt begann es mit der Einführung von Maschinen im Produktionsprozess (Industrie 1.0), ging über mit der Nutzung von Elektrizität als Antriebskraft für diese Maschinen (Industrie 2.0) und gelangte danach zur starken Nutzung von Computern und Rechentechnik im Rahmen der industriellen Prozesse (Industrie 3.0). Bei den jeweiligen Ausbaustufen war es zudem notwendig, dass sich auch angrenzende Bereiche entsprechend mitentwickeln, z.B. die technischen Wissenschaften, aber auch Politik und Gesellschaft. Der Begriff der Revolution passt für die genannten drei Ausbaustufen, weil er die ständigen Umbrüche und Entwicklungsschübe gut einfängt. Die Fachliteratur ist sich jedoch nicht einig, ob man ebenso von einer Revolution bei der Industrie 4.0 sprechen kann.

Die Befürworter sehen als Ausgangspunkt für die Entwicklung die Etablierung der Mechatronik. Sie ist interdisziplinär aufgestellt und befasst sich mit Methoden und Ansätzen aus den folgenden Disziplinen: Mechanik, Maschinenbau, Elektronik, Elektrotechnik, Informatik und Informationstechnik (\cite{ragavan2016engineering}, S. 2). Daneben gibt es eine scheinbar ständig wachsende Liste weiterer angrenzender Disziplinen, z.B. Optoelektronik. Bei der Mechatronik und Industrie 4.0 steht der Produktionsprozess im Vordergrund und über die Zeit folgte eine immer weiter wachsende Übernahme neuer Entwicklungen aus den einzelnen Disziplinen, um diesen immer weiter zu optimieren. In diesem Zusammenhang wird der Begriff \textit{Cyber-Physical Systems} (CPS) häufig genannt. Es gibt hierzu verschiedene Definitionsansätze in der Literatur. Im Folgenden wird CPS definiert als ein "Mechatronisches System, mit enger Integration von physischen, digitalen und kommunikativen Komponenten" (\cite{ragavan2016engineering}, S. 3). Bei CPS kommt es zu einer stetig wachsenden Verschmelzung von physischen und digitalen Systemen. Nach \cite{klein2019architektur} spielt CPS bei Digitalen Zwillingen eine wichtige Rolle.

Die Kritiker der Begriffswahl Industrie 4.0 entgegnen, dass der Begriff eine deutsche Wortschöpfung ist und sich das Gesagte auf die deutsche Wirtschaft bezieht. In anderen Regionen der Welt wurden im gleichen Zeitraum ähnliche Initiativen gegründet\footnote{Industry IoT Consortium (\url{https://www.iiconsortium.org/about-us/}) oder siehe \cite{trotta2018industry}, S. 2 ff}. All diese Initiativen unterscheiden sich in ihren thematischen Ausrichtungen, auch wenn es teils große Überlappungen gibt. Daher kann man nicht von einer Revolution im Sinne der Industrie 1.0 - 3.0 sprechen.

\subsection{Life Cycle Assessment (LCA)}

Bei der Lebenszyklusanalyse (Life Cycle Assessment) handelt es sich um "eine systematische Analyse potenzieller Umweltwirkungen und der Energiebilanz von Produkten während des gesamten Lebensweges" (Zitat Wikipedia \footnote{https://de.wikipedia.org/w/index.php?title=Lebenszyklusanalyse\&oldid=234859970}).

[...]

\subsection{Schlüsselwörter}

Die folgende Aufzählung enthält alle Schlüsselwörter, mit denen eine Ontologie in Teilen assoziiert sein muss, um für diese Arbeit relevant zu sein. Die Schlüsselwörter gehören zu sich überschneidenden Themengebieten und lassen sich teilweise nicht voneinander eindeutig abgrenzen.

\begin{enumerate}
    \item \textbf{Industry 4.0} - Dieser Begriff bezeichnet die vierte industrielle Revolution bezogen auf den technologischen Fortschritt. In einem I4.0 Szenario steht der Fertigungsprozess (Manufacturing-Prozess) im Mittelpunkt, dieser stellt die Hauptaktivität dar bei der Techniken und Hilfsmittel eingesetzt werden wie, autonome Roboter, Cyber-Physical-Systems, Big-Data Analysis und IoT.
    \item \textbf{Smart Manufacturing} (sowie Smart Factory oder Factory 4.0) - Ein Oberbegriff für eine (evolutionäre) Initiative zur Änderung der Geschäftsstrategie im Fertigungsbereich. Hierbei sollen Fertigungszentren modernisiert werden unter Einbeziehung von Technologien wie Industry Internet of Things (IIOT), 3D-Modeling, Industrieautomation, Mobile-Computing und Intelligente analysebasierte Entscheidungssysteme.
    \item \textbf{Big Data + Data Analytics} - TODO
    \item \textbf{Cyber-Physical Systems} (CPS) - Cyberphysische Systeme (engl. "Cyber-Physical Systems") sind Systeme, bei denen informations- und softwaretechnische mit mechanischen Komponenten verbunden sind, wobei Datentransfer und -austausch sowie Kontrolle bzw. Steuerung über ein Netzwerk, wie das Internet, in Echtzeit erfolgen. Wesentliche Bestandteile sind mobile und bewegliche Einrichtungen, Geräte und Maschinen (darunter auch Roboter), eingebettete und vernetzte Systeme (Internet der Dinge). Sensoren registrieren und verarbeiten Daten aus der physikalischen Welt, Aktoren (Antriebselemente) wirken auf die physikalische Welt ein, sodass z.B. Weichen gestellt, Schleusen geöffnet, Fenster und Türen geschlossen, Produktionsvorgänge begonnen, geändert und angehalten werden \cite{GablerCPS2024}.
    \item \textbf{Supply Chain Management} - TODO
    \item \textbf{Life Cycle Assessment} (Nachhaltigkeitsbewertung) - Relevant; TODO?
\end{enumerate}

Alle Ontologien, die mit einem oder mehreren der genannten Schlüsselwörter aufbauen, basieren womöglich auf anderen Ontologien, Ontologie-Design-Pattern oder Vokabularen. In diesem Fall werden auch diese mit in die Arbeit aufgenommen, auch wenn ihnen keine der obigen Schlüsselwörter zugeordnet wurden.

\subsection{Life Cycle Assessment (LCA)}

Bei der Lebenszyklusanalyse (Life Cycle Assessment) handelt es sich um "eine systematische Analyse potenzieller Umweltwirkungen und der Energiebilanz von Produkten während des gesamten Lebensweges" (Zitat Wikipedia \footnote{https://de.wikipedia.org/w/index.php?title=Lebenszyklusanalyse\&oldid=234859970}).

\subsection{Internationale Standards ISIC, NACE und CPC}

Im Kontext dieser Arbeit sind die folgenden internationalen Standards von Interesse:

\begin{itemize}
    \item \textbf{International Standard Industrial Classification} (ISIC): Diese ist die U.N. Klassifikation von wirtschaftlichen Aktivitäten. Diese liegt derzeit in der Revision 4 vor und der hauptsächliche Anwendungszweck ist die international einheitliche Einordnung wirtschaftlicher Aktivitäten um das Erstellen aussagekräftiger Wirtschaftsstatistiken zu ermöglichen, die international vergleichbar sind \cite{UNSDISIC}.
    \item \textbf{die Statistische Systematik der Wirtschaftszweige in der Europäischen Gemeinschaft} (französisch Nomenclature statistique des activités économiques dans la Communauté européenne, kurz NACE): Die NACE ist eine auf den europäischen Raum angepasste Version der ISIC und wird gleichfalls für die Analyse von Wirtschaftsdaten im europäischen Raum verwendet \cite{EurostatNACE}.
    \item \textbf{Central Product Classification} (CPC): Ist eine Klassifikation der U.N., welche derzeit in der Version 2.1 vorliegt. Hierbei werden Güter und Produkte hauptsächlich anhand physischer Form und ihrer industriellen Herkunft. Im letzteren Punkt ist die CPC sehr stark mit der ISIC gekoppelt \cite{UNSDCPCv21}.
\end{itemize}

Hierbei handelt es ich um Werke, die einzeln oder im Verbund ein Framework darstellen, um internationale Vergleichbarkeit von Waren, Gütern oder auch wirtschaftlichen Aktivitäten zu ermöglichen. Sie spielen für diese Arbeit auf der Ebene der Begriffe eine Rolle. [...]

TODO: Nennung passender Definitionen [...]

TODO: Überlege/Prüfe, ob langfristig relevant für das Dokument

\section{Verwandte Arbeiten (Related Work)}

- Welche Untersuchungen wurden bereits dahingehend unternommen?

- Wozu gab es noch keine oder nur unzureichende Untersuchungen?

\subsection{Ontologies for Industry 4.0}
%
% Ontologies for Industry 4.0
% https://www.cambridge.org/core/services/aop-cambridge-core/content/view/BF86BB5310356D642C82470D67974804/S0269888919000109a.pdf/div-class-title-ontologies-for-industry-4-0-div.pdf

Zu Beginn der Publikation "Ontologies for Industry 4.0" \cite{kumar2019ontologies} geben die Autoren zuerst eine kurze Einordnung des Themengebietes Industrie 4.0, Factory 4.0 und Smart Manufacturing.
Diese wird ergänzt durch eine historische Einordnung.
Danach folgt eine Zusammenfassung relevanter Herausforderungen (z.B. Mensch-Maschine-Kommunikation oder Datenanalyse) und die genutzten Technologien (z.B. Internet of Things, 5G, Cloud), um diese zu bewältigen.
Im zweiten Teil der Arbeit wird zuerst der ontologische Rahmen vorgestellt.
Die Autoren listen eine Reihe von Anwendungsbereichen für Ontologien auf, jedoch ohne diese näher einzuordnen.
Immerhin wird am Ende des Abschnitts ein kurzer Vergleich zwischen zwei verschiedenen Strukturierungsansätzen für Ontologien andiskutiert.
Im Kapitel 2.2 stellen die Autoren einige Resultate ihrer Standardisierungsvorhaben vor und beschreiben den aktuellen Stand.
Das Hauptaugenmerk lag dabei zuerst auf der Kommunikation zwischen Robotern, menschlichen Bedienern, Endkunden und diversen Zulieferern.
Weiterhin spielt die Automatisierung eine große Rolle.
Laut Aussage der Autoren sollten diese Themen zuerst behandelt werden, um daraus resultierende Probleme zu vermeiden.
Ihre Arbeit basiert auf dem IEEE-Beitrag namens "IEEE 1872-2015 Standard Ontologies for Robotics and Automation"\footnote{1872-2015 - IEEE Standard Ontologies for Robotics and Automation: \url{https://ieeexplore.ieee.org/document/7084073}}.
Diese Ontologien bieten eine ontologische Basis für weiterführende Konzepte aus dem Bereich Industrie 4.0.
Danach folgt eine kurze Vorstellung der folgenden Ontologien:
\begin{enumerate}
    \item \textbf{CORA:} Core Ontology for Robotics and Automation
    \item \textbf{ROA:} The Ontology for Autonomous Robotics\footnote{In der zugehörigen Publikation\cite{olszewska2017ontology} kürzen die Autoren die Ontologie jedoch mit ORA ab.}
    \item \textbf{ORArch}: Ontology for Robotic Architecture
    \item \textbf{O4I4}: Ontology for Industry 4.0
\end{enumerate}

Im dritten Teil der Arbeit gehen die Autoren die folgenden Szenarien im Bereich Industrie 4.0 ein: Smart-rapid prototyping scenario und UAV’s good delivery scenario.
Jedes Szenario wird kurz eingeführt und danach der Einsatz der Ontologien beschrieben.
Es ist erwähnenswert, dass die Publikation an mehreren Stellen eine Auflistung von relevanten Publikationen enthält, die jedoch nur benannt aber nicht inhaltlich eingeordnet werden. Damit bieten die Autoren einen guten Ausblick, jedoch fehlt dem Leser die Einordnung in das Papier.






\subsection{Where to Publish and Find Ontologies? A Survey of Ontology Libraries}

Die Autoren des Papiers "Where to Publish and Find Ontologies? A Survey of Ontology Libraries" \cite{d2012publish} (Jahr 2012) geben einen Überblick über Ontologie-Bibliotheken.
Diese Publikation ist für diese Arbeit relevant, weil sie einerseits einen Überblick vorhandener Ontologie-Bibliotheken gibt und andererseits für die Einordnung von Ontologien nützliche Beiträge liefert.

Es wird ebenfalls von den Autoren festgestellt, dass die Vorgängerarbeiten Systeme behandelten, von denen die meisten nicht mehr existieren bzw. nicht weiterbetrieben werden.

Neben der Nennung von verschiedenen Ontologie Bibliotheken stellen die Autoren ebenfalls Charakteristiken zur Einordnung solcher Systeme vor (u.a. Zwecke und thematische Abdeckung).

[...] noch viele Seiten u.a. zu den Funktionen der einzelnen Portale, Qualitätsanforderungen/Gatekeeping etc.

\subsubsection{Nicht erreichbare oder eingestellte Ontologie-Bibliotheken}

\textbf{Überlegung: vielleicht Liste an zentraler Stelle einsetzen? Oder als als CSV ergänzen?}

Die folgenden Ontologie-Bibliotheken sind nicht mehr erreich- bzw. nutzbar:

\begin{enumerate}
    \item Cupboard (\url{https://cupboard.open.ac.uk:8081/cupboard-search/}, Link von \url{https://www.w3.org/wiki/Ontology_repositories\#Cupboard})
    \item OntoSelect (\url{https://olp.dfki.de/OntoSelect/}, Link aus Publikation \cite{buitelaar2008ontology})
    \item ONTOSEARCH2 (Keine URL, weil implementiert als Java Framework, Primärpublikation \cite{pan2006ontosearch2})
    \item Schema-Cache (\url{https://schemacache.test.talis.com/}, Link von \url{http://vocamp.org/wiki/Where_to_find_vocabularies\#SchemaCache})
    \item TONES Ontology Repository (\url{http://owl.cs.manchester.ac.uk/repository/}, Link von \url{https://www.w3.org/2001/sw/wiki/TONES})
\end{enumerate}

\subsubsection{Nutzbare bzw. erreichbare Ontologie-Bibliotheken}

\textbf{Überlegung: vielleicht Liste an zentraler Stelle einsetzen? Oder als als CSV ergänzen?}

Die folgenden Ontologie-Bibliotheken sind weiterhin benutzbar:

\begin{enumerate}
    \item BioPortal (\url{https://bioportal.bioontology.org/}, Themengebiete: Biomedizin)
    \item OBO Foundry (\url{https://obofoundry.org/}, Themengebiete: Biologie und Biomedizin)
    \item oeGOV (\url{http://www.oegov.us/}, Themengebiet: e-Government)
    \item Ontology Lookup Service (\url{https://www.ebi.ac.uk/ols4}, Themengebiete: Biomedizin)
    \item Ontology Design Patterns (\url{http://ontologydesignpatterns.org/wiki/Main\_Page}, viele Themengebiete, siehe auch \url{http://ontologydesignpatterns.org/wiki/Community:Domain})
    \item ONKI ontology server (\url{https://onki.fi/en/}, verschiedene Themengebiete)
\end{enumerate}


- "Researchers have used many different names to refer to systems for collecting ontologies
and making them available: ontology directory, ontology repository, ontology library,
ontology archive. We will use the term ontology library to refer to these systems. We
broadly define an ontology library as a Web-based system that provides access to an
extensible collection of ontologies with the primary purpose of enabling users to find and
use one or several ontologies from this collection" (TOPDO einordnen)





\subsection{Industry Portal}

Industry Portal: Ontologies for industry https://industryportal.enit.fr/ontologies
- Nutzen die folgenden Kategorien zur thematischen Einordnung (http://industryportal.enit.fr/landscape)
-+ Computer Science, systems and electrical engineering
-+ Material Science and Engineering
-+ Mechanical and Industrial Engineering
-+ Physics and Chemistry
-+ Thermal and Process Engineering

=> Verknüpfe mit (Where to publish ... Paper) \cite{d2012publish}
Konkret: Einordnung als "\textbf{ontology library} as a Web-based system that provides access to an
extensible collection of ontologies with the primary purpose of enabling users to find and
use one or several ontologies from this collection"



\subsection{Bioportal}

- Thematische Einordnung von Ontologien anhand einer Kategorie-Hierarchie
  - Abfragbar über REST mittels: https://data.bioontology.org/categories

=> Verknüpfe mit (Where to publish ... Paper) \cite{d2012publish}
Konkret: Einordnung als "\textbf{ontology library} as a Web-based system that provides access to an
extensible collection of ontologies with the primary purpose of enabling users to find and
use one or several ontologies from this collection"


\section{Methodik}

Wie Recherche nach Ontologien?
Worauf wurde bei Aufnahme einer Ontologie konkret geachtet?

TODO


\section{Ontologie und relevante Unterscheidungskriterien}

- Interessante Einteilung von Ontologien in >> Ontologies for transportation research: A survey, Seite 3, https://eil.mie.utoronto.ca/wp-content/uploads/2015/06/katsumi-TRC81.pdf
-+ bieten praktische Def. was eine Ontologie ist
-+ Integrationsansatz zwischen Ontologien und anderen Datenformen wird erläutert (auch verwenden!)
-+ S. 3 hat schöne Abstufung von Termen --- bis Common Logic (Ontologic Spectrum)
-+ S. 5 ff. Kriterien für Einordnung von Ontologien: z.B. Granularität, Evaluation, ...
-+ man stellt weiterhin relevante Klassen für den Bereich Transport auf und prüft, inwieweit sich diese in den untersuchten Ontologien wiederfinden
-+ Interessant ist, dass bei jeder Ontologie auch eine kleine Übersicht der Themen bzw. Ober und Unterklassen beigefügt ist


Zum Thema was ist eine Ontologie und was nicht. Wo ziehen wir die Grenze? Beispiele: PDF mit Text und Begriffen VS. OWL Datei

* Ontologien haben je nach Fachgebiet eine abweichende Definition (siehe die folgenden Unterpunkte). Aus diesem Grund wird sich im Folgenden auf die Ontologien beschränkt, die mindestens in Beschreibungslogik oder FOL vorliegen. Wichtig ist hier das Kriterium der formalen Modellierung. Ontologien werden dabei definiert als: " as artifacts that allow formal modelling of the entities and the relations in a system and are expressed in a formal, machine-readable format that computers can process" (Quelle: \textit{Gangemi, Aldo, Nicola Guarino, Claudio Masolo et al. 2002. “Sweetening ontologies with DOLCE”: Proceedings of the 13th European Conference on Knowledge Engineering and Knowledge Management 2473: 166-181. DOI: 10.1007/3-540-45810-7\_18.})

-+ https://www.isko.org/cyclo/ontologies1.jpg: Vorschlag für Einteilung von Ontologien nach ihrer Komplexität???

-+ Semantic Stair case, von Olensky 2010: https://www.isko.org/cyclo/ontologies3.jpg (https://www.isko.org/cyclo/ontologies, Punkt 3.1) VS. Levels of ontological precision: https://www.isko.org/cyclo/ontologies4.jpg



- bitte prüfen, ob das Paper und die Inhalte hier noch Sinn machen >>> 3  Ontology-Method: Developing Consistent and Modular Software Models with Ontologies:  https://www.researchgate.net/profile/Axel-Cyrille-Ngonga-Ngomo/publication/221026467\_Developing\_Consistent\_and\_Modular\_Software\_Models\_with\_Ontologies/links/02e7e52b00bd23ebb9000000/Developing-Consistent-and-Modular-Software-Models-with-Ontologies.pdf
 + Untersuchung und Benutzung einer Ontologie aus Sicht eines Softwareentwicklers
 + man propagiert eine Aufspaltung des Softwaremodells (worüber die Software in einer Domäne verankert wird) in ein Konzeptuales und ein Domainmodel
 + Vorschlag für Aufteilung des Softwaremodells in eine Task Ontologie, Domain Ontologie und eine Top Level Ontologie
   - Task Ontologie: Repräsentiert die Inhalte, auf die Software zugreift, um ihre Aufgaben erfüllen zu können
   - Domain Ontologie: Repräsentiert die Inhalte der Domänen in der die Software agiert
   - Top Level Ontologie: Integriert Task und Domain Ontologie, und bildet die Basis worüber weitere Domain Ontologien angebunden/integriert werde können

- Unterscheidung zwischen Kern- und Domainontologie? und Top Level Ontologie nochmal tun


- \textbf{(eigenes Papier)} Ontologies for Industry 4.0 \cite{kumar2019ontologies}
  + Begriff Industrie 4.0 wird erläutert (S. 1)
  + S. 2: "Everything related to the production could be represented in the cyberspace, from the smallest and least significant raw material or component up to the complete product and all the machinery
  involved in its production (Rosen et al., 2015)" => guter Aufhänger für uns?
  + Frage: Sollte das Papier im Sinne von Industrie 4.0 geschrieben werden??
  + S. 2: Im Rahmen von I4.0 werden Daten von Zulieferern, Kunden und Fabriken gesammelt und evaluiert, bevor sie zur Produktion herangezogen werden. Man verspricht sich davon flexiblere und anpassbarere Produktionsprozesse und in Echtzeit umgestellt werden können.
  + vor I4.0 nutzte man scheinbar den Ansatz nach ISA-95 (5-Stufen-Model)



- European Materials \& Modelling Ontolog (EMMO) https://github.com/emmo-repo/ + https://emmc.eu/news/emmo-1-0-0-alpha-release/

\section{Relevante Ontologien}

[...]

TODO: binde finale Version der Grafik ontology-map.png hier ein

Die Ontologien werden im Folgenden noch kurz vorgestellt.
Weiterführende Metadaten zu den Ontologien finden Sie in unseren Rohdaten, welche in Form einer CSV-Datei öffentlich zur Verfügung stehen.

\textbf{TODO Link}

Sie finden darin weitere Informationen zu den jeweiligen Ontologien, z.B. Link zum Projekt oder den RDF/OWL-Dateien.

\subsection{Top Level Ontologien (upper ontology)}

\subsubsection{Suggested Upper Merged Ontology (SUMO)}

Die SUMO\footnote{Projektseite: \url{https://www.ontologyportal.org/index.html}} Ontologie ist eine Top Level Ontologie und deren Entwicklung wird laut Projektseite von Adam Pease (technical editor) geleitet.
Die Ontologie ist in der Sprache Standard Upper Ontology Knowledge Interchange Format (SUO-KIF) geschrieben\footnote{\url{https://github.com/ontologyportal/sigmakee/blob/master/suo-kif.pdf}} und
bestand anfangs aus Ontologie-Beiträgen auf einer Mailing-Liste, die jedoch mit der Zeit stark erweitert wurden (\cite{niles2001towards}, S. 2 ff.).
SUMO stellt Definitionen für sehr allgemeine Begriffe bereit, welche für aufbauende Kernontologien wiederverwendet werden können.
Mit diesen Begriffen werden häufig verwendete Teile unserer Realität beschrieben, z.B. Objekte und Prozesse.
Laut eigener Aussage wird sie für die Forschung und Entwicklung von Anwendungen aus den Bereichen Suche, Linguistik und Reasoning verwendet\footnote{Zitat "\textit{The Suggested Upper Merged Ontology (SUMO) and its domain ontologies form the largest formal public ontology in existence today. They are being used for research and applications in search, linguistics and reasoning.}" von Projektseite: \url{https://www.ontologyportal.org/index.html}}.
Auf \url{https://github.com/ontologyportal/sumo} findet man die KIF-Dateien des Projektes, welche die einzelnen Ontologie-Bestandteile repräsentieren (z.B. Anatomy.kif, Military.kif).
Das Projekt scheint weiterhin gepflegt zu werden, der letzte Commit auf Github ist ca. 3 Wochen alt\footnote{\url{https://github.com/ontologyportal/sumo/commit/69f22dc041ae5f307279134968a740ff74c04ff2}}.

\subsection{Ontology Design Pattern}

Im Folgenden eine Auflistung aller Ontology Design Pattern

\subsubsection{Ontology Design Pattern für LCA}

% PDF: https://ceur-ws.org/Vol-1461/WOP2015_pattern_abstract_1.pdf

Die Publikation "A Minimal Ontology Pattern for Life Cycle Assessment Data"\cite{vardem2015anminimal} stellt ein Ontology Design Pattern vor, welches als Grundlage für umfangreichere Ontologien aus dem Bereich LCA genutzt werden kann.
Ziel hinter dem Ontology Design Pattern war es, die semantische Interoperabilität zwischen LCA-Datensätzen zu stärken.
Konkret geht es um die Einigung bei der zentralen Nomenklatur rund um LCA.
Bisher bestand nach Aussage der Autoren eine Lücke, wodurch die Datenmodelle nicht durch Konzeptuelle Modelle gestützt wurden (\cite{vardem2015anminimal}, S. 2 ff.).
Darauf aufbauende Arbeiten hätten damit das Problem, dass man bei deren Nutzung zu fehlerhaften oder nicht reproduzierbaren Ergebnisse kommen könnte.
Insbesondere bei der Modellierung von Datensätzen ist eine semantische Interoperabilität wichtig, um z.B. Vergleichbarkeit zu gewährleisten.
Am Ende wird ein Kohlekraftwerk als Modellierungsbeispiel vorgestellt.

Die zugehörige  OWL-Datei lässt sich leider nicht mehr über die in der Publikation genannte URL aufrufen\footnote{URL führt zu HTTP-404-Fehler: \url{https://descartes-core.org/ontologies/lca/1.0/LCAPattern.owl}}.

\subsubsection{GeoLink Ontology Design Pattern}

% PDF: https://ceur-ws.org/Vol-1486/paper_99.pdf
% PDF: https://ebiquity.umbc.edu/_file_directory_/papers/763.pdf

Die GeoLink Ontology Design Pattern \cite{krisnadhi2015geolinkStart} sind Teil des GeoLink-Projektes, dessen Zielstellung es war, ein erweitertes Daten-Integrations- und Discovery-Framework unter Nutzung von Semantic Web Technologien für das EarthCube Programm bereitzustellen.
Das EarthCube-Programm ist eine, durch die NSF (National Science Foundation) gesponsorte, Initiative zur Verbesserung der digitalen Infrastruktur der Geowissenschaften (Schwerpunkt Ozeanographie).

Es wurde eine hohe Diversität in den verschiedenen Datensätzen festgestellt, z.B. aufgrund verschiedener Modellierungsansätze und genutzter Vokabulare.
Aus diesem Grund sahen die Autoren davon ab, eine komplexe Ontologie zu entwickeln (und viele Integrationsprobleme zu riskieren). Stattdessen entwickelte man verschiedene Ontology Design Pattern, welche eigenständig und damit modular einsetzbar sein sollen.
Zusammen mit anderen Bestandteilen ermöglichen sie die Erstellung komplexerer Ontologien.

==> \textbf{TODO:} leider erfährt man aus den oben genannten GeoLink Papieren nichts zu den in \cite{vardem2015anminimal}, S. 4 erwähnten Property Pattern.

Diese Arbeit wurde in die Untersuchung aufgenommen, weil gewisse Bestandteile in der "Ontology Design Pattern für LCA" verwendet werden (\cite{vardem2015anminimal}, S. 4).

\subsubsection{material transformation pattern}

\cite{vardem2015anminimal} verweißt auf S. 4 auf material transformation pattern

https://scholar.google.de/scholar?q=An+ontology+design+pattern+for+material+transformation.\&hl=de\&as\_sdt=0\&as\_vis=1\&oi=scholart



\subsection{Ontologien und Vokabulare}

TODO: Kurze Erklärung warum beides hier?

\subsubsection{BONSAI Ontology (Big Open Network for Sustainability Assessment Information)}

Die BONSAI Ontologie ist im Themenbereich Life Cycle Sustainability Assessment (LCSA) verankert, der sich aus ökologischem Life Cycle Assessment (LCA) und ökonomischem Life Cycle Costing (LCC) zusammensetzt (\cite{ghose2022core}, S. 2).
BONSAI ist die Abkürzung für Big Open Network for Sustainability Assessment Information.
Die Ontologie wurde so entworfen, dass sie in verschiedenen Bereichen der industriellen Ökologie einsetzbar ist, ohne dabei zu stark von einem Framework abhängig zu sein.
Sie baut dabei auf der Publikation "A Minimal Ontology Pattern for Life Cycle Assessment Data" \cite{vardem2015anminimal} auf und nutzt die darin vorgeschlagene Nomenklatur für LCA.
Der Grund für die Entwicklung der Ontologie war die Verbesserung der semantischen Interoperabilität zwischen verschiedenen LCA-Datensätzen.
Für die Modellierung von Messwerten und Maßeinheiten wird auf die Ontology of Units of Measure \cite{rijgersberg2013ontology} zurückgegriffen.

Das Projekt wirkt verwaist und wird nur sehr selten gepflegt. Im Ontologie-Repository auf Github ist der letzte Commit über zwei Jahre alt\footnote{\url{https://github.com/BONSAMURAIS/ontology}}. Im zugehörigen Repository für sonstige Entwicklungsbeiträge und Diskussionen ist der letzte Commit über vier Jahre alt\footnote{\url{https://github.com/BONSAMURAIS/BONSAI-ontology-RDF-framework}}. Der Zustand der Issue-Tracker stützt diesen Eindruck.

\subsubsection{Car Options Ontology}

wird genutzt von Volkswagen Vehicles Ontology (https://data.industryportal.enit.fr/ontologies/VVO/submissions/1/download?apikey=019adb70-1d64-41b7-8f6e-8f7e5eb54942)

\subsubsection{CORA Ontology}

% https://repositorio.ipcb.pt/bitstream/10400.11/2815/1/proceedings_IROS2014_workshop_IEEE_ORA.pdf#page=13

Die Core Ontology for Robotics and Automation (abgekürzt: CORA) \cite{prestes2014core} wurde von der IEEE RAS Ontology for Robotics and Automation Working Group (IEEE RAS ORA WG) entwickelt und ist ein Ansatz zur Standardisierung der vorhandenen Robotik-Terminologie.
Diese Entwicklung war Teil einer Initiative zur Schärfung der Terminologien u.a. in den folgenden Themengebieten: Industrieroboter (Industrial Robotics), Roboter-assistierte Chirugie (Surgical Robotics) and Service Roboter (Service Robotics).
Mit CORA soll definiert werden, was ein Roboter ist, daher werden u.a. die Konzepte Roboter (robot), Robotisches System (robotic system) und Bestandteil eines Roboters (robotic part) näher beschrieben.
Im Rahmen der Initiative wurden weitere Ontologien entwickelt, die auf CORA aufbauen.
CORA nimmt dabei eine Vermittlerrolle ein, sodass darauf aufbauende Ontologien konsistente Konzepte verwenden können, was zu langfristiger Kompatibilität beitragen kann.
Die Ontologie baut dabei der auf der Top Level Ontologie SUMO auf, wobei weitere Unterontologien von CORA entwickelt wurden, um semantische Lücken zwischen SUMO und CORA zu überbrücken (siehe Unterkapitel für CORAX, POS und RPARTS).

Auf der Projektseite \url{https://github.com/srfiorini/IEEE1872-owl} findet man die OWL Spezifikation von CORA und anderen Ontologien hinter dem IEEE 1872-2015 Standard\footnote{\url{https://standards.ieee.org/ieee/1872/5354/}}. Der letzte Commit im Repository ist vom 21.10.2020\footnote{\url{https://github.com/srfiorini/IEEE1872-owl/commit/ffb6613350b28c3a710f8daf2a1e9e953ded7ed8}}, weshalb man annehmen kann, dass das Projekt nicht mehr aktiv betreut wird.

\subsubsection{CORAX Ontology}

% PDF: https://www.sciencedirect.com/science/article/am/pii/S0736584514000659

Die CORAX Ontologie wurde im Rahmen der CORA Ontologie entwickelt, mit dem Ziel notwendige Konzepte zu definieren, die für die CORA zu allgemein waren, aber in der Top Level Ontologie SUMO nicht existierten (\cite{fiorini2015extensions}, S. 3).
Dazu gehören Definitionen zu den Konzepten Design (design, im Sinne von Produktdesign), physische Umgebung (physikal environment), Interaktion (interaction) und Künstliches System (artifical system).
So wurde zum Beispiel zwischen SUMO:Artificat und CORA:Robotic System das Konzept CORAX:Artificial System (CORAX) eingeführt (\cite{fiorini2015extensions}, S. 6 ff.).

Auf der Projektseite \url{https://github.com/srfiorini/IEEE1872-owl} findet man neben der OWL Spezifikation von CORA auch die von CORAX. Der letzte Commit im Repository ist vom 21.10.2020, weshalb man annehmen kann, dass das Projekt nicht mehr aktiv betreut wird.

\subsubsection{IOF Supply Chain Reference Ontology}

Die Supply-Chain-Reference-Ontology ist eine Domänenontologie, welche im Umfeld der Industry-Ontology-Foundation (IOF) entwickelt und gepflegt wird. Ihre Kompetenzen sind Standardisierung, Semantische Mediation, wie z.B. Datenkonversion und -integration, sowie automatisiertes Inferieren und Reasoning hinsichtlich der Lieferkettenverfolgung (Traceability Use-Case) und Lieferantenermittlung (Supplier-Discovery), die primären Andwendungsfälle dieser Ontologie \cite{Wallace2021}. Sie wurde von der Supply-Chain-Working-Group an der Texas-State-University unter der Leitung von Dr. Farhad Ameri\footnote{\url{https://oagi.org/pages/supply-chain-working-group}} entwickelt. Dieses Projekt wurde letztmalig am 10. Oktober 2023 aktualisiert und scheint daher aktiv gepflegt zu werden. Die Version 1 der Ontologie, wurde am 1. Januar 2023 veröffentlicht. Die Ontologie ist gekoppelt an die IoF-Core-Ontology\cite{Kulvatunyou2022} und daher als Mid-Level Ontologie einzuordnen. 


\subsubsection{GoodRelations Vokabular}

Das GoodRelations\footnote{https://www.heppnetz.de/ontologies/goodrelations/v1.html} Vokabular ist laut eigener Aussage ein Vokabular zur Modellierung Daten im E-Commerce Bereich (z.B. Produkte, Dienstleistungen oder Unternehmen).
Es wurde von Martin Hepp u.a. mit dem Ziel entwickelt, Produkte und Dienstleistungen auf E-Commerce-Webseiten mit zusätzlichen Metadaten auszuzeichnen, um die Sichtbarkeit in Suchmaschinen zu erhöhen.
GoodRelations wurde 2012 als E-Commerce Kern in das Vokabular schema.org aufgenommen\footnote{\url{http://blog.schema.org/2012/11/good-relations-and-schemaorg.html}}.
Damit gingen alle Klassen und Properties in das schema.org Vokabular über und wurden darin integriert\footnote{Beispiel: Klasse Product \url{https://schema.org/Product}}.
Für Zugriff auf die maschinenlesbaren Daten bitte im Kapitel für schema.org schauen.

\subsubsection{LCA scope ontology}

\cite{vardem2015anminimal} verweißt auf S. 4 auf LCA scope ontology

https://www.acsu.buffalo.edu/~yhu42/papers/2015\_stscope\_ontology.pdf

\subsubsection{Ontology for Autonomous Robotics (ROA / ORA)}
TODO lesen: https://www.aolivaresalarcos.com/pdf/2017-roman.pdf
cite \cite{olszewska2017ontology}

\subsubsection{Ontology for Industry 4.0 (O4I4)}

\cite{kumar2019ontologies}

Grafik machen: O4I4 nutzt CORA, ROA/ORA und ORArch (\cite{kumar2019ontologies}, S. 7)

\subsubsection{Ontology of units of Measure (OM)}

Die Ontology of units of Measure (OM) wurde von dem Wissenschaftler Hajo Rijgersberg im Rahmen seiner Tätigkeit in der Abteilung Computer-Science der Universität Wageningen (Niederlande) entwickelt, als integrativer Teil einer größeren Ontologie\cite{rijgersberg2013ontology}, der Ontology of Quantitative Research (OQR).
Die OM Ontologie versucht die Konzepte der Domäne von Maßen und Einheiten umfassend abzubilden, welche in Teilen bei anderen Ontologien fehlen. Dies betrifft z.B. die Konzepte Quantity und Prefix. Die OM Ontologie bezieht sich dabei in Teilen auf die QUDT Ontologie und kommt in verschiedenen Ontologie-Projekten, wie z.B dem Bioportal\cite{OM_BioPortal}, oder innerhalb der Bonsai-Ontologie \cite{ghose2022core} zum Einsatz.

Auf der Projektseite \url{https://github.com/HajoRijgersberg/OM} steht die Ontologie in der Version 2.0 zur Verfügung. Die Ontologie wird aktiv betreut, so ist z.B. der letzte Commit im Repository vom 08.02.2024.

\subsubsection{Position Ontology (POS)}

% PDF: https://www.sciencedirect.com/science/article/am/pii/S0736584514000659

Die Position Ontology (POS) wurde als Erweiterung der SUMO Ontologie im Rahmen des CORA Projektes entwickelt (\cite{fiorini2015extensions}, S. 17 ff).
Sie komplementiert die CORA Ontologie mit neuen Konzepten zur Positionierung und Orientierung, was insbesondere bei der Modellierung von Informationen über Roboter und ihre Umgebung wichtig ist.
Dafür wurden u.a. die folgenden Konzepte eingeführt: position, PositionCoordinateSystem, PositionPoint.
Diese und andere Konzepte und Properties erlauben die Modellierung von Positionen in einem Koordinatensystem sowie die relative Positionierung.

Auf der Projektseite \url{https://github.com/srfiorini/IEEE1872-owl} findet man neben der OWL Spezifikation von CORA auch die von POS. Der letzte Commit im Repository ist vom 21.10.2020, weshalb man annehmen kann, dass das Projekt nicht mehr aktiv betreut wird.

\subsubsection{Product Ontology}

http://www.productontology.org/ (nutzt GoodRElations)

[...]

\subsubsection{Quantities, Units, Dimensions and Types ontology (QUDT)}

Unit-Measure-Ontologien stellen eine konzeptuelle Repräsentation von Quantitäten , Quantitätsarten, Einheiten dar. Der bekannteste und weitläufig eingesetzte Vertreter, ist die von der NASA entwickelte QUDT\cite{QUDTOntology}. Innerhalb dieser Ontologie werden im Kern die Konzepte physikalischer Quantitäten. Quantitätsarten, Einheiten und Dimensionen, sowie Datentypen repräsentiert. Die Ontologie basiert auf dem internationalen Standard der SI-Unit (franz. Système international d'unités) und liegt im OWL-Format vor. Ein Hauptanwendungszeck einer solchen Ontologien, ist das Konvertieren von physikalischen  Einheiten unter Berücksichtigung der Dimensionalität.
QUDT wird eingesetzt in Luft- und Raumfahrt, Industrieproduktion, sowie Energie- und Wasserversorgung. Ein Beispiel ist die Verwendung von QUDT innerhalb der Systems-Architecture-Ontology (SAO) für Cyber-Physical-Systems.

TODO: Projektseite und letzte Aktivität

\subsubsection{RPARTS Ontology}

% PDF: https://www.sciencedirect.com/science/article/am/pii/S0736584514000659

Die RPARTS Ontology ist eine Teil-Ontologie der CORA Ontology, welche zur Modellierung von Bestandteilen von Robotern verwednet werden kann. (\cite{fiorini2015extensions}, S. 14 ff).
Der zugrundeliegende Ansatz ist, dass Roboter (\textit{devices}) immer aus Bestandteilen (anderen \textit{devices}) bestehen.
Man unterscheidet zwischen Roboter-Bestandteilen zur Kommunikation (\textit{Robot communicating part}), Verarbeitung (\textit{Robot processing part}), Sensorik (\textit{Robot sensing part}) und Handlung (\textit{Robot actuating part}).
[...]?

Auf der Projektseite \url{https://github.com/srfiorini/IEEE1872-owl} findet man neben der OWL Spezifikation von CORA auch die von POS. Der letzte Commit im Repository ist vom 21.10.2020, weshalb man annehmen kann, dass das Projekt nicht mehr aktiv betreut wird.

\subsubsection{schema.org Vocabulary}

Schema.org\footnote{\url{https://www.schema.org}} ist eine von Firmen und Freiwilligen gepflegte quelloffene Ontologie (bzw. Vokabular) zur Auszeichnung von Daten auf Webseiten.
Das Ziel für dessen Entwicklung ist es, Suchmaschinen zusätzliche Metadaten über die Webseiteninhalte bereitstellen zu können.

\subsubsection{Vehicle Sales Ontology (VSO)}

Quelle: https://www.heppnetz.de/ontologies/vso/ns

Wird genutzt von VVO (IndustryPortal)

nutzt GoodRelations Vokabular

\subsubsection{Volkswagen Vehicles Ontology (VVO)}

Aufgrund fehlender Originalquelle (z.B. wiss. Publikation oder Projektseite) und der unterschiedlichen Dokumentation zur Volkswagen Vehicles Ontology, werden die aktuellen Funde im Folgenden einzeln dargelegt.

Die Ontologie wird in einer Untersuchung namens "Case Study: Contextual Search for Volkswagen and the Automotive Industry" des W3C aus dem Jahre 2011 erwähnt (\cite{greenly2011case}, S. 2).
In der Studie ging es u.a. um die Verbesserung der Suche für Nutzer in digitalen Quellen.
In diesem Zuge wurde u.a. die Volkswagen Vehicles Ontology entwickelt, um Volkswagen-spezifische Konzepte bei Fahrzeugen beschreiben zu können.
Der Autor der Ontologie war Martin Hepp.
Man nutzte diese und andere Vokabulare/Ontologien, um die eigenen Daten anzureichern, um kontextbasierte Suchanfragen zu ermöglichen.
Die URL \url{http://purl.org/vvo/ns} wird als Quelle für die Ontologie angegeben, jedoch gelangt man darüber am Ende nur zu \url{https://www.volkswagen.co.uk/en.html}, welches keine weiteren Informationen bereitstellt.
Weitere Details zur Ontologie selbst wurden nicht genannt.

Der Link \url{https://industryportal.enit.fr/ontologies/VVO} führt zu einem Projekteintrag auf IndustryPortal.
Als Autor wird Abdelouadoud Rasmi angegeben, der jedoch keine ersichtliche Verbindung zu Volkswagen hat.
In der zugehörigen OWL-Datei\footnote{https://data.industryportal.enit.fr/ontologies/VVO/submissions/1/download?apikey=019adb70-1d64-41b7-8f6e-8f7e5eb54942} findet man jedoch im Abschnitt Autor auch Martin Hepp und bei Beitragende (contributors) u.a. den Erstautor der oben genannten W3C Studie William Greenly.
Aufgrund dieses und anderer Gemeinsamkeiten, z.B. gleicher Name (Volkswagen Vehicles Ontology) und XML-Namensraum (http://purl.org/vvo/ns\#), wird angenommen, dass es sich um die gleiche Ontologie handelt. Die Ontologie auf IndustryPortal hat als Veröffentlichungsdatum den 2. November 2022, wobei als Veröffentlichungsdatum bei der Version der 12. Oktober 2010 genannt wird.
Es liegen keine Informationen vor, dass zwischen diesen Daten weitere Veränderungen an der Ontologie vorgenommen wurden.
Ich gehe daher davon aus, dass es sich hier lediglich um einen erneuten Upload der Version von 2010 handelt.

Für die Ontologie existiert auch auf DataScientia ein Eintrag unter \url{http://liveschema.eu/dataset/lov\_vvo}.
Als Autor wird wieder Martin Hepp angegeben.
Der Eintrag wurde jedoch am 3. Februar 2020 erstellt und repräsentiert auch die Version 1.0 vom 12. Oktober 2010.


\section{Fazit und weitere Arbeiten}

[...]

TODO: Erwähnung KI-Werk Projekt



\medskip

\printbibliography

\end{document}

\typeout{get arXiv to do 4 passes: Label(s) may have changed. Rerun}